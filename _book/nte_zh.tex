\documentclass[twoside,openright,headings=optiontohead]{ctexbook} %{scrbook} %
\renewcommand{\baselinestretch}{1.3}  %行間距倍率
\columnsep 7mm
%\renewcommand\thepage{}
\usepackage{setspace}


\usepackage[
b5paper=true,
%CJKbookmarks,
unicode=true,
bookmarksnumbered,
bookmarksopen,
hyperfigures=true,
hyperindex=true,
pdfpagelayout = SinglePage,
%pdfpagelayout = TwoPageRight,
pdfpagelabels = true,
pdfstartview = FitV,
colorlinks,
pdfborder=001,
linkcolor=black,
anchorcolor=black,
citecolor=black,
pdftitle={Nothing to Envy},
pdfauthor={Barbara Demick},
pdfsubject={爸爸三定律},
pdfkeywords={爸爸},
pdfcreator={lzf}
]{hyperref}


\usepackage{graphics,graphicx,pdfpages}
\usepackage{caption} %用于取消标题编号 \caption*{abc}



\usepackage{xeCJK}
\providecommand{\tightlist}{%
   \setlength{\itemsep}{0pt}\setlength{\parskip}{0pt}}
   
\usepackage{indentfirst}
\setlength{\parindent}{2.0em}

%正文字体
\setCJKmainfont[Path=fonts/,
BoldFont={fzfsjt.ttf},
ItalicFont={fzssjt.ttf}, %方正书宋简体
BoldItalicFont={fzfsjt.ttf},
SlantedFont={fzfsjt.ttf},
BoldSlantedFont={fzfsjt.ttf},
SmallCapsFont={fzfsjt.ttf}
]{fzfsjt.ttf}
\setCJKsansfont[Path=fonts/]{fzfsjt.ttf}
\setCJKmonofont[Path=fonts/]{fzfsjt.ttf}
\setmainfont[Path=fonts/]{fzfsjt.ttf}
\setsansfont[Path=fonts/]{fzfsjt.ttf}
\setmonofont[Path=fonts/]{fzfsjt.ttf}
% Icon 字体
\newfontfamily{\FA}[Path=fonts/]{FontAwesome.otf} %License: SIL OFL 1.1
\newfontfamily{\EA}[Path=fonts/]{EyesAsia-Regular.otf} %License: MIT
\newfontfamily{\EN}[Path=fonts/]{SourceSansPro-ExtraLightIt.otf} %License: SIL OFL 1.1

% 頁面及文字顏色
\usepackage{xcolor}
\definecolor{TEXTColor}{RGB}{50,50,50} % TEXT Color
\definecolor{PinYinColor}{RGB}{180,180,180} % TEXT Color
\definecolor{NOTEXTColor}{RGB}{0,0,0} % No TEXT Color
\definecolor{BGColor}{RGB}{240,240,240} % BG Color
\definecolor{Gray}{RGB}{246,246,246}


\usepackage{multicol}

\makeindex
\renewcommand{\contentsname}{{数学家谈怎样学数学}}
% \usepackage{fancyhdr} % 設置頁眉頁腳
% \pagestyle{fancy}
% \addtolength{\headwidth}{\marginparsep}
% %\addtolength{\headwidth}{\marginparwidth}
% \fancyhf{} % 清空當前設置
% \renewcommand{\headrulewidth}{0pt}  %頁眉線寬,設為0可以去頁眉線
% \renewcommand{\footrulewidth}{0pt}  %頁眉線寬,設為0可以去頁眉線


\usepackage{titleps}% http://ctan.org/pkg/{titleps,lipsum}
\newpagestyle{newstyle}{
  \setheadrule{.4pt}% Header rule
  \sethead[\scriptsize {\FA \ }数学家谈怎样学数学]% even left
    []% even centre
    [{\tiny{\textcolor{Gray}{\FA \ }}}\thepage]% even right
    {{\tiny{\textcolor{Gray}{\FA \ }}}\thepage}% odd left
    {}% odd centre
    {\scriptsize {\FA \ }数学家谈怎样学数学}% odd right
}

\pagestyle{newstyle}



\usepackage{titletoc}
\dottedcontents{section}[100em]{\bfseries}{100em}{100em} % 去掉目录虚线

\usepackage{geometry}
\geometry{b5paper, left=2cm,right=2cm,top=2.5cm,bottom=2cm,foot=2.5cm}
%\usepackage[b5paper,tmargin=2.5cm,bmargin=2.5cm,lmargin=3.5cm,rmargin=2.5cm]{geometry}
\newcommand{\Icon}{\fontsize{600pt}{\baselineskip}\selectfont}

\begin{document}
	\frontmatter
%	\begin{figure}[ht]
%		\begin{center}
%			\includepdf[height=\paperheight]{images/cover.jpg}
%		\end{center}
%	\end{figure}
\newpage
{\color{TEXTColor}
	%\begin{multicols}{2}
		\tableofcontents
	%\end{multicols}
	\newpage
	\mainmatter
% \fancyhead[LO]{{\scriptsize {\FA \ } 数学家谈怎样学数学}}%奇數頁眉的左邊
% %\renewcommand{\headrulewidth}{0pt} % optional
% %\fancyhead[L]{\nouppercase{\leftmark}}
% \fancyhead[RO]{{\tiny{\textcolor{Gray}{\FA \ }}}\thepage}
% \fancyhead[LE]{{\tiny{\textcolor{Gray}{\FA \ }}}\thepage}
% \fancyhead[RE]{{\scriptsize {\FA \ }\leftmark}}%偶數頁眉的右邊
% \fancyfoot[LE,RO]{}
% \fancyfoot[LO,CE]{}
% \fancyfoot[CO,RE]{}

\mainmatter

\hypertarget{ux5b66ux4e60ux548cux7814ux7a76ux6570ux5b66ux7684ux4e00ux4e9bux4f53ux4f1a}{%
\chapter*{学习和研究数学的一些体会}\label{ux5b66ux4e60ux548cux7814ux7a76ux6570ux5b66ux7684ux4e00ux4e9bux4f53ux4f1a}}
\addcontentsline{toc}{chapter}{学习和研究数学的一些体会}

人贵有自知之明。我知道,我对科学研究的了解不全面。也知道,搞科学极重要的是独立思考,各人应依照各人自己的特点找出最适合的道路。听了别人的学习、研究方法,就以为我也会学习研究了,这个就无异于吃颗金丹就会成仙,而无需经过勤修苦练了。

今天把我五十年来的经验教训,所见所闻、所体会的给你们介绍,目的在于尽可能把我的经验作为你们的借鉴,具体问题具体分析,具体的个人应当想出最适于自己的有效方法来。

\textbf{我第一点准备和同志们谈的问题是速度、是效率}。速度是实现社会主义现代化的保证。例如说象我这样又老又拐的人,我在前头走你们赶我不费劲,一赶就赶上,而我要赶你们,除非你们躺下来睡大觉,否则我无论如何是赶不上的。现在世界上科学发展很快,我们如果没有超过美国的速度和效率就不可能赶上美国。我们没有超过日本的速度和效率,我们就不可能赶上日本。如果我们的速度仅仅和美、日等国一样,那么也只能是等时差的赶,超就是一句空话。所以说,我们应当首先在速度和效率上超过他们。

但要我们的速度和效率超过他们有没有可能呢?这似乎是一个大问题,其实不然,我在美国呆过,在英国呆过,也在苏联呆过。我看到他们的速度不是神话般的快不可及。我们是赶得上超得过的!我们许多美籍华人,如果他们的速度不能超过一般的美国人的话,也就不会成为现代著名的科学家了。所以事实证明,只要我们努力下功夫,赶超是完全可以的。就以我自己来说,我是三六年到英国的,在那里呆了两年,回国后在昆明乡下住了两年,四零年就完成了堆垒素数论的工作。后来五零年回国后,在五八年之前,我们的数论、代数、多复变函数论等等都达到了世界上的良好的水平。所以经验告诉我们纯数学的一门学科,有四五年就能在世界上见头角了。你们现在时代更好了,党中央一举粉碎了``四人帮'',带来了科学的春天。在这样的条件下边,我敢断言,只要肯下功夫,努力钻研,只要不浪费一分一秒的时间,我们是能够赶上世界先进水平的。特别是我们数学,前有熊庆来、陈建功、苏步青等老前辈的榜样,现在又有许多后起之秀,更多的后起之秀也一定会接踵而来。\\
\textbf{二、消化} 抢速度不是越级乱跳,不是一本书没有消化好就又看一本,一个专业没有爬到高处就又另爬一个山峰。我们学习必须先从踏踏实实地读书讲起,古时候总说这个人``博闻强记''``学富五车''。实际上古人的这许多话到现在已是不足为训了。五车的书,从前是那种大字的书,我想一个指甲大小的集成电路就可以装它五本十本,学书五车,也不过十几块几十块集成电路而已。现在也有相似的看法,说某人念了多少多少书,某人对世界上的文献记的多熟多熟,当然这不是不必要的,而这只能说走了开始的第一步,如果不经过消化,实际抵不上一个图书馆,抵不上一个电子计算机的记忆系统。人之所以可贵就在于会创造,在于善于吸收过去的文献的精华,能够经过消化创造前人所没有的东西。不然人云亦云世界就没有发展了,懒汉思想是科学的敌人,当然也是社会发展的敌人。

什么是消化?检验消化的最好的方法就是``用''。会用不会用,不是说空话,而是在实际中考验。碰到这个问题束手无策,碰到那个问题又是一筹莫展,即使他能写几篇模仿性的文章,写几本抄抄译译的稿著,这同社会的发展又有什么关系呢?当然我不排斥初学的人写几篇模仿性的文章,但绝不能局限于此,须发皆白还是如此。

消化,只有消化后,我们才会灵活运用。如果社会主义建设需要我们,我们就会为社会主义建设服务,解决问题,贡献力量。客观的问题上面不会贴上标签的,告诉你这需要用数论,那个是要用泛函,而社会主义建设所提出来的问题是各种各样无穷无尽的,想用一个方法套上所有的实际问题,那就是形而上学的做法。只有经过独立思考和认真消化的学者,才能因时因地根据不同的问题,运用不同的方法真正解决问题。

当然,刚才说消化不消化只有在实际中进行检验。但是同学们不一定就有那么多的实践机会,在校学习的时候有没有检查我们消化了没有的方法呢?我以前讲过,学习有一个由薄到厚,再由厚到薄的过程。你初学一本书,加上许多注解,又看了许多参考书,于是书就由薄变厚了。自己以为这就是懂了,那是自欺欺人,实际上这还不能算懂。而真正懂,还有一个由厚到薄的过程。也就是全书经过分析,扬弃枝节,抓住重点,甚至于来龙去脉都一目了然了,在没有这条定理前,人家是怎样想出来的,这样才能说是开始懂了,这也是一个检验自己是否消化了的方法。当然,这个方法不如前面那种更踏实。总的一句话,检验我们消化没有,弄通没有的最后标准是实践,是能否灵活运用解决问题。也许有人会说这样念书太慢了。我的体会不是慢了,而是快了。因为我们消化了我们以前念过的书,再看另一本书时,我们脑子里的记忆系统就会排除那些过去弄懂了的东西。而只注意新书中自己还没有碰到过的新东西。所以说,这样脚踏实地的上去,不是慢了而是快了。不然的话囫囵吞枣的学了一阵,忘掉一阵,再学再忘,白费时光事小,使自己``于国于家无望''事大。更可怕的是好高骛远。例如中学数学没学懂,他已读到大学三、四年级的课程,遇到困难,但又不屑于回去复习,再去弄通中学的东西,这样前进,就愈进愈糊涂,陷入泥坑,难于自拔。有时候阅读同一水平的书,如果我们以往的书弄懂了消化了,那么在同一水平书里找找以往书上没有的东西就可以过去了。找不到很快送上书架,找到一点两点就只要把这一两点弄懂就得了,这样读书就快了,不是慢了。

读书得法了,然后看文献,实际上看文献和看书没有什么不同,也是要消化。不过书上是比较成熟的东西,去粗取精,则精多粗少。而文献是刚出来的,往往精少而粗多。当然不排除有些文章,一出来就变成经典著作的情况。但这毕竟是少数的少数。不过多数文章通过不多时间就被人们遗忘了。有了吸取文献的基础,就可以搞研究工作。

这里我还要强调一下独立思考。独立思考是搞科学研究的根本,在历史上,重大的发明没有一个是不通过独立思考就能搞出来你的。当然这并不等于说不接受前人的成就而``独立''``思考''。例如有许多人,搞哥德巴赫猜想,对前人的工作一无所知,这样搞,成功的可能性是很小的。独立思考也并不是说不要攻书,不要看文献,不要听老师的讲述了。书本、文献、老师都是要的,但如果拘泥于这些,就会失去创造力,使学生变成教师的一部分,这样就会愈缩愈小,数学上出了收敛的现象。只有独立思考才能够跳出这个框框,创造出新的方法,创造出新的领域,推动科学的进步。独立思考不是说一个人独自在那里冥思苦想,不和他人交流。独立思考也要借助别人的结果,也要依靠群众和集体的智慧。独立思考也可以补救我们现在导师不够,导师经验较差,导师太忙顾不过来,但这都需独立思考来补救。甚至于象我们过去在昆明被封锁的时候,外国杂志没处来,我们还是独立思考,想出新的东西来,而想出来的东西和外国人并重复。即使有,也别怕。例如说,我青年时在家里发表过几篇文章,而退稿的很多,原因是别人说你的这篇文章那本书里已有此定理了,那篇文章在某书里也已有证明了等等。面对这种情况是继续干呢?还是就泄气呢?觉得上不起学,老是白费时间搞前人所搞过的东西。当时,我并没有这样想,在收到退稿时反而高兴,这是我明白,原来某大科学家搞过的东西,我在小店里也能搞出来。因此我还是加倍继续坚持搞下去了。我这里并不是说过去的文献不要看,而是说即使重复了人家的工作也不要泄气,要对比一下自己搞出来的同已有的有什么区别,是不是他们的比我们的好,这样就学习人家长处,就有进步,如果我们还有长处就增加了信心。

我们有了独立思考,没有导师或文献不全,就都不会成为我们的阻力。相反,有导师我们也还要考虑考虑讲的话对不对,文献是否完整了\ldots{}\ldots{}。总之,科学事业是善于独立思考的人所创造出来的,而不是象我前面所说的等于几块集成电路的那种人创造出来的,因为这种人没有创造性。毛主席指出:研究问题,要由此及彼,由表及里,去粗取精,去伪存真。做到这四点,就非靠独立思考不可,不独立思考就只得其表,取其粗,只能够伪善杂存,无法明辨是非。

三、搞研究工作的几种境界

1.照葫芦画瓢的模仿。模仿性的工作,实际上就等于做一个习题。当然,做习题是必要的,但是一辈子做习题而无新创又有什么意思呢?

2.利用成法解决几个新问题。这个比前面就进了一步,但是我们在这个问题上也应区别一下。直接利用成法也和做习题差不多,而利用成法,又通过一些修改,这就走上搞科学研究的道路了。

3.创造方法,解决问题。这就更进了一步。创造方法是一个重要的转折,是自己能力的提高的重要表现。
4.开辟方向。这就更高了,开辟了一个方向,可以让后人做上几十年,成百年。这对科学的发展来讲就是有贡献。我是粗略地分为以上这四种,实际上数学还有许多特殊性的问题。象著名问题你怎样改进它,怎样解决它,这在数学方面一般也是受到称赞的。在二十世纪初希尔伯特提出了二十三个问题。这许多问题,有些是会对数学的本质产生巨大的影响。费尔马问题我想这是大家都知道的。这个问题如用初等数论方法解决了,那没有发展前途,当然,这样他可以获得``十万马克''。但对数学的发展是没有多大意义的。而Kummer虽没有解决费尔马问题,但他为研究费尔马却创造了理想数,开辟了方向。现在无论在代数、几何、分析等方面,都用上了这个概念,所以它的贡献远比解决一个费尔马问题大。所以我觉得,这种贡献就超过了解决个别难题。

我对同志们提一个建议,取法乎上得其中,取法乎中得其下。研究工作还有一条值得注意的,要攻得进去,还要打得出来。攻进去需要理论,真正深入到所搞专题的核心需要理论,这是人所共知的。可是要打得出来,并不比钻进去容易。世界上有不少数学家攻是攻进去了·但是进了死胡同就出不来了·这种情况往往使其局限在一个小问题里,而失去了整个时间。这种研究也许可以自娱,而对科学的发展和社会主义的建设是不会有作用的。

\textbf{四、我还想跟同学们讲一个字,``漫''}

我们从一个分支转到另一个分攴是把原来所搞分支丢掉跳到另一分支吗?如果这样就会丢掉原来的。而``漫''就是在你搞熟弄通的分支附近,扩大眼界,在这个过程中逐渐转到另一分支,这样,原来的知识在新的领域就能有用,选择的范围就会越来越大。我赞成有些同志钻一个问题钻许多年搞出成果,我也赞成取得成果后用漫的方法逐步转到其它领域。

鉴别一个学问家或个人,一定要同广,同深联系起来看。单是深,固然能成为一个不坏的专家,但对推动整个科学的发展所起的作用,是微不足道的。单是广,这儿懂一点,那儿懂一点,这只能欺欺外行,表现表现他自已博学多才,而对人民不可能做出实质性的成果来。

数学各个分支之间,数学与其它学科之闻实际上没有不可邀越的鸿沟。以往我们看到过细分割,各搞一行的现象,结果呢?哪行也没搞好。所以在钻研一科的同时,把与自己学科或分支相近的书和文献浏览浏览,也是大有好处的。

\textbf{五、我再讲一个``严''字}

不单是搞科学研究需要严,就是练兵也都要从难,从严。至于说相互之间说好听的话,听了谁都高兴。在三国的时候就有两个人,一个叫孔融,一个叫祢衡,祢衡吹捧孔融是仲尼复生。孔融吹捧祢衡是颜回再世。他们虽然相互捧的上了九宵云外,而实际上却是两个饭桶,其下场都被曹操直接或间接地杀死了。当然,听好话很高兴而说好话的人也有他的理论,说我是在鼓励年青人。但是这样的鼓励,有的时候不仅不能把年青人鼓励上去,反而会使年青人自高自大,不再上进。特别是若干年来,我知道有许多对学生要求从严的教师受到冲击。而一些分数给的宽,所谓关系搞得好的,结果反而得到一些学生的欢迎。这种风气只会拉社会主义的后腿。到现在我们要一个老师对我们要求严格些,而老师都不敢真正对大家严格要求。所以我希望同学们主动要求老师严格要求自己,对不肯严格要求的老师,我们要给他们做一定的思想工作,解除他们的顾虑。同样一张嘴,说几句好听的话同说几句严格要求的话,实在是一样的,而且说说好听话大家都欢迎,这有何不好呢?并且还有许多人认为这样是团结好的表现。若一听到批评,就认为不团结了,需要给他们做思想工作了等等。实际上这是多余的,师生之间的严格要求,只会加强团结,即使有一时想不开的地方,在长远的学习、研究过程中,学生是会感到严师的好处的。同时对自己的要求也要严格。大庆三老四严的作风,我们应随时随地、人前人后地执行。
我上面谈到过的消化,就是严字的体现,就是自我严格要求的体现。一本书马马虎虎的念,这在学校里还可以对付,但是就这样毕了业,将来在工作中间要用起来就不行了。我对严还有一个教训,在1964年,我刚走向实践想搞一点东西的时候,在乌蒙磅礴走泥丸的地方,有一位工程师,出于珍惜国家财产的心情,就对我说:``雷管现在成品率很低,你能不能降低一些标准,使多一些的雷管验收下来。''我当时认为这个事情好办。我只要略略降低一些标准,验收率就上去了。但后来在梅花山受到了十分深刻的教训。使我认识到,降低标准1\%,实际就等于要牺牲我们四位可爱的战士的生命。这是我们后来搞优选法的起点。因为已经造成了的产品,质量不好,我们把住关,把废品卡住,但并不能消除由于废品多而造成的损失。如果产品质量提高了,废品少了,那么给国家造成的损失也就自然而然地小了。我这并不是说质量评估不重要,我在1969年就提倡过。不过我们搞优选法的重点就在这里。这就和治病、防病一样,以防为主。搞优选法就是防止次品出现。而治就是出了废品进行返工,但这往往无法返工,成为不治之症。老实说,以往我对学生的要求是习题上数据错一点没有管,但是自从那次血的教训,使我得到深刻的教育。我们在办公室里错一个1\%好像不要紧,可是拿到生产、建设的实践中去,就会造成极大的损失。所以总的一句话,包括我在内,对严格要求我们的人,应该是感谢不尽的。对给我们戴高帽子的人,我也感谢他,不过他这顶帽子我还是退还回去,请他自己戴上。同学们,求学如逆水行舟,不进则退。只要哪一天不严格要求自己,就会出问题。当然,数学工作者,从来没有不算错过题的。我可以这样说一句,天下只有哑巴没有说过错话;天下只有白痴没有错过问题;天下没有数学家没算错过题的。错误是难免要发生的,但不能因此而降低我们的要求,我们要求是没有错误,但既然出现了错误,就应该引以为教训。不负责任的吹嘘,虽然可能会使你高兴,但我们要善于分析,对这种好说恭维话的人要敬而远之。不负责任地恭维人,是旧社会遗留下来的恶习,我们要尽快地把它洗刷掉。当然,别人说我们好话,我们不能顶回去,但我们的头脑要冷静、要清醒,要认识到这是顶一文钱不值的高帽子,对我的进步毫无益处。

实事求是,是科学的根本。如果搞科学的人不实事求是,那就搞不了科学,或就不适于搞科学。党一再提倡实事求是的作风,不实事求是地说话、办事的人,就背离了党的要求。科学是来不得半点虚假的。我们要正确估价好的东西,就是一时得不到表扬.也不要灰心,因为实践会证明是好的。而不太好的东西,就是一时得到大吹大擂,不会多久也就会烟消云散了。我们要有毅力,要善于坚持。但是在发现是死胡同的时候,我们也得善于转移,不过发现死胡同是不容易的,不下功夫是不会发现的。就是退出死胡同时,也得搞清楚它死在何处,经过若干年后,发现难点解决了,死处复活了,我就又可以打进去。失败是经常的事,成功是偶然的。所有发表出的成果,都是成功的经验,同志们都看到了,而同志们哪里知道,这是总结了无数失败的经验教训才换来的。跟老师学习就有这样一个好处,好老师可以指导我们减少失败的机会,更快吸收成功的经验,在这个基础上又创造出更好的东西。还可以看到他的失败的经验,和山穷水尽疑无路柳暗花明又一村地从失败又怎样转到成功的经验,切不可有不愿下苦功侥幸成功的想法。天才,实际上在他很漂亮地解决问题之前是有一个无数次失败的艰难过程。所以同学们千万别怕失败,千万别以为我写了一百张纸了,但还是失败了,我搞一个问题已两年了,而还没有结果等就丧失信心,我们应总结经验,找到我们失败的原因,不再重复我们失败的道路。总的一句话,失败是成功之母。

似懂非懂,不懂装懂比不懂还坏。这种人在科学研究上是无前途的,在科学管理上是瞎指挥的。如果自己真的知己和承认不懂,则容易听取群众的意见,分析群众的意见,尊重专家的意见,然后和大家一起做出决定来,\ldots{}..。特别对你们年青人,没有经过战火的考验(战火的考验是最好的考验,错误的判断就打败仗,甚至于被敌人消灭),也没有深入钻研的经验,就不知道旁人的甘苦。如果没有组织群众性的搞科学研究的锻炼和能力,就必然陷入瞎指挥的陷阱。虽然他(或她)有雄心想办好科学,实际上会造成拆台的后果。所以我要求你们年青人有两条:1、有对科学钻深钻懂一行两行的锻炼。2、能有搞科学实验运动,组织群众,发动群众,把科学知识普及给群众的本领。不然,对四个现代化来说就会起拉后腿的作用。对个人来说一事无成,而两鬓已斑。

当前在两条不可得兼的时候,择其一也可,总之没有农民不下田就有大丰收的事情,没有不在机器边能生产出产品的工人。脑力劳动也是如此,养得肠肥脑满,清清闲闲,饱食终日无所用心的科学家或科学工作组织者是没有的。单凭天才的科学家也是没有的,只有勤奋,才能勤能补拙,才能把天才真正发挥出米。天资差的通过勤奋努力,就可以赶上和超过有天才而不努力的人。古人说,人一能之己十之,人十能之己百之,这是大有参考价值的名言。

\textbf{六、要善于暴露自己}

不懂装懂好不好?不好!因为不懂装懂就永远不会懂。要敢于把自己的缺点和不懂的地方暴露出来,不要怕难为情。暴露出来顶多受老师的几句责备,说你``连这个也不懂'',但是受了责备后不就懂了吗?可是不想受责备,不懂装懂,这就一辈子也不懂。科学是实事求是的学问,越是有学问的人,就越是敢暴露自己,说自己这点不清楚,不清楚经过讨论就清楚了。在大的方面,百家争鸣也就是如此,每家都敢于暴露自己的想法,每家都敢批评别人的想法,每家都接受别人的优点和长处,科学就可以达到繁荣、昌盛。``''四人帮''搞得大家对问題表态不好,不表态也不好,明知不对也不敢暴露,这样就自然产生僵化,僵化是科学的死敌,科学就不能发展。不怕低,就怕不知底。能暴露出来,让老师知道你的底在哪里,就可以因才施教。同时,懂也不要装着不懂。老师知道你懂了很多东西,就可以更快地带着你前进。也就是一句话,懂就说懂,不懂就说不懂,会就说会,不会就说不会,这是科学的态度。

好表现。这似乎是一个坏事,实际也该分析一下,如果自己不了解,或半知半解而就卖弄他的渊博,一这是真正的好表现,这不好。而把自己懂的东西交流给旁人,使别人以更短的时间来掌握我们的长处,这种表现是我们欢迎的,这不是好表现,这是好表现。科学有赖于相互接触,互相交流彼此的长处,这样我们就可以兴旺发达。

我上面所讲的有片面性,更重要的是为人民服务的回题。大家政治理论学习比我好,同时我们这里也没有时间了,就不在这里多讲了。我用一句话结束我的发言。\\
不为个人,而为人民服务。

当然我这篇讲话就是这个主题,但没能充分发挥,不过人贵有自知之明,我对这方面的认识更弱于我对数学的认识了,而政治干部比我搞业务的人就知道的更多,我也就不想在这里超出我的范围多说了。

\hypertarget{ux6570ux5b66ux7684ux7528ux573aux4e0eux53d1ux5c55}{%
\chapter*{数学的用场与发展}\label{ux6570ux5b66ux7684ux7528ux573aux4e0eux53d1ux5c55}}
\addcontentsline{toc}{chapter}{数学的用场与发展}

数(读作shu 四声)起源于数(读作shu三声),如一、二、三、四、五\ldots{}\ldots{},一个、两个、三个\ldots{}\ldots{}。量起源于量。先取一个单位标准,然后一个单位一个单位地
量。天下虽有各种不同的量(各种不同的量的单位如尺、斤、 斗、秒、伏特、欧姆和卡路里等等),但都必须通过数才能确
切地把实际的情况表达出来。所以``数''是各种各样不同量的 共性,必须通过它才能比较量的多寡,才能说明量的变化。

''量''是贯穿到一切科学领域之内的,因此数学的用处也就渗透到一切科学领域之中。凡是要研究量、量的关系、量
的变化、量的关系变化、量的变化的关系的时候,就少不了数学。不仅如此,量的变化还有变化,而这种变化一般也是
用量来刻划的。例如,速度是用来描写物体的变化的动态的。而加速度则是用来刻划速度的变化。量与量之间有各种各样的关系,各种各样不同的关系之间还可能有关系。为数众多的关系还有主从之分---也就是说,可以从一些关系推导出另一些关系来。所以数学还研究变化的变化,关系的关系,共性的共性,循环往复,逐步提高,以至无穷。

数学是一切科学得力的助手和工具。它有时由于其他科学的促进而发展,有时也先走一步,领先发展,然后再获得
应用。任何一门科学缺少了数学这一项工具便不能确切地刻划出客观事物变化的状态,更不能从已知数据推出未知的数
据来,因面就减少了科学预见的可能性,或者减弱了科学预见的精确度。

恩格斯说:``纯数学的对象是现实世界的空间形式和数量关系。``数学是从物理模型抽象出来的、它包括数与形两方面
的内容。以上只提要地讲了数最关系,现在我们结合宇宙之大来说明空间形式。

\hypertarget{ux5b87ux5b99ux4e4bux5927}{%
\section*{宇宙之大}\label{ux5b87ux5b99ux4e4bux5927}}
\addcontentsline{toc}{section}{宇宙之大}

宇宙之大,宇宙的形态,也只有通过数学才能说得明白。天圆地方之说,就是古代人民用几何形态来描绘客观宇宙的
尝试。这种``苍天如圆盖,陆地如棋局''的宇宙形态的模型,后来被航海家用事实给以否定了。但是,我国从理论上对这
一模型提出的怀疑要早得多,并且也同样的有力。论点是:``混沈初开,乾坤始奠,气之轻清上浮者为天,气之重浊者
下凝者为地。''但不知轻清之外,又有何物?也就是圆盖之外,又有何物?三十三天之上又是何处?要想解决这样的问
题,就必须借助于数学的空间形式的研究。

四维空间听来好象有些神秘,其实早已有之,即以``宇宙''二字来说,``四方上下曰宇,往古来今曰宙,以喻天地''
(《准南子·原道》)。就是宇是东西、南北、上下三维扩展的空间,而宙是一维的时间。牛顿时代对宇宙的认识也就是如
此。宇宙是一个无边无际的三维空间,而一切的日月星辰都安排在这框架中运动。找出这些星体的运动规律是牛顿的一
大发明。也是物理模型促进数学方法,而数学方法则是用来说明物理现象的一个好典范。由于物体的运动:不是等加速
度,要描绘不是等加速度,就不得不考虑速度时时在变化的情况。于是既创造了新工具------微积分,又发现了万有引力
定律。有了这些,宇宙间一切星辰的运动初步统一地被解释了。行星凭什么以椭圆轨道绕日而行的,何时以怎样的速度,
达到何处等,都可以算出来了。

有人说西方文明之飞速发展是由于欧几里得几何的推理方法和进行系统实验的方法。牛顿的工作也是逻辑推理的一
个典型。他用简单的几条定律推出整个的力学系统.大至解释天体的运行,小到造房、修桥、杠杆、称物都行。但是人们
在认识自然界而建立的理论总是不会一劳永逸完美无缺的,牛顿力学不能解释的问题还是有的。用它解释了行星绕日公
转,但行星自转又如何解释呢?地球自转一天24小时有昼有夜。还有一个有名的问题:水星进动每百年42".这是牛顿
力学无法解释的。

爱因斯坦不再把``宇''、``宙''分开来看,也就是时间也在进行着。每一瞬间三维空间中的物质在占有它一定的位置。
他根据麦克斯韦-洛伦兹的光速不变假定,并继承了牛顿的相对性原理而提出了狭义相对论。狭义相对论中的洛伦兹变换
把时空联系在一起,当然并不是消灭了时空特点。如向东走三里,再向西走三里,就回到原处,但时间则不然,共用了
走六里的时间。时间是一去不复返地流逝着。值得指出的是有人推算出狭义相对论不但不能解释水星进动问题,而且算
出的结果是``退动''。这是误解。我们能算出进动28'',即客观数的三分之二。另外,有了深刻的分析,反而能够浅出,
连微积分都不要用,并且在较少的假定下,就可以推出爱因斯坦狭义相对论的全部结果。

爱因斯坦进一步把时、空、物质联系在一起,提出了广义相对论。用它可以算出水星进动是43'',这是支持广义相
对论的一个有力证据。由于证据还不多,因此对广义相对论还有不少看法,但它的建立有赖于数学上的先行一步。如先有
了黎曼几何。另一方面它也给数学提出了好些到现在还没有解决的问题。对宇宙的认识还将有多么大的进展,我不知
道,但可以说,每一步都是离不开数学这个工具的。

\hypertarget{ux7c92ux5b50ux4e4bux5fae}{%
\section*{粒子之微}\label{ux7c92ux5b50ux4e4bux5fae}}
\addcontentsline{toc}{section}{粒子之微}

佛经上有所谓``金栗世界'',也就是一粒粟米也可以看作一个世界。这当然是佛家的幻想。但是我们今天所研究的原
子却远远地小于一粒栗米,而其中的复杂性却不亚于一个太阳系。

即使研究这样小的原子核的结构也还是少不了数学。描述原子核内各种基本粒子的运动更是少不了数学。能不能用
处理普遍世界的方法来处理核子内部的问题呢?情况不同了!在这里,牛顿的力学,爱因斯坦的相对论都遇到了困难。在
目前人们应用了另一套数学工具。如算子论,群表示论,广义函数论,固体子等。这些工具都是近代的产物。即使如此,
也还是不能完整地说明它。

在物质结构上不管分子论、原子论也好,或近代的核子结构、基本粒子的互变也好,物理科学上虽然经过了多次的
概念革新,但自始至终都和数学分不开。不但今天,就是将来,也有一点是可以肯定的,就是一定还要用数学。

是否有一个统一的处理方法,把宏观世界和微观世界统一在一个理论之中,把四种作用力统一在一个理论之中,这
是物理学家当前的重大问题之一。不管将来他们怎样解决这个问题,但是在处理这些问题的数学方法必须统一。必须有
一套既可以解释宏观世界又可以解释微观世界的数学工具。数学一定和物理学刚开始的时候一样,是物理科学的助手和
工具。在这样的大间题的解决过程中,也可能如牛顿同时发展天体力学和发明微积分那样促进数学的新分支的创造和形
成。

\hypertarget{ux706bux7badux4e4bux901f}{%
\section*{火箭之速}\label{ux706bux7badux4e4bux901f}}
\addcontentsline{toc}{section}{火箭之速}

在今天用``一日千里''来形容慢则可,用来形容快则不可了!人类可创造的物体的速度远远地超过了``一日千里''。飞
机虽快到日行万里不夜,但和宇宙速度比较,也显得缓慢得很。

不妨回忆一下,在星际航行的开端---由诗一般的幻想进入科学现实的第一步,就是和数学分不开的。早在牛顿时
代就算出了每秒钟近八公里的第一字宙速度,这给科学技术工作者指出了奋斗目标。如果能够达到这一速度,就可以发
射地球卫星。1970年我国发射了第一颗人造卫星,东方红的歌声传遍了全世界。数学工作者自始至终都参与这一工作(当
然,其中不少工作者不是以数学工作者见称,而是运用数学工具者)。作为人造行星环绕太阳运行所必须具有的速度是
11.2公里/秒,称为第二宇宙速度;脱离太阳系飞向恒屋际空间所必须具有的速度是16.7公里/秒,称为第三宇宙速度。
这样的目标,也将会逐步去实现。

顺便提一下,如果我们宇宙航船到了一个星球上,那儿也有如我们人类一样高级的生物存在。我们用什么东西作为
我们之间的媒介。带幅画去吧,那边风景殊,不了解。带一段录音去吧,也不能沟通。我看最好带两个图形去。一个``数,''
一个``数形关系''(勾股定理)(图1和图2)。

为了使那里较高级的生物知道我们会几何证明,还可送去下面的图形。即``青出朱入图''(图3)。这些都是我国古代
数学史上的成就。

\hypertarget{ux5316ux5de5ux4e4bux5de7}{%
\section*{化工之巧}\label{ux5316ux5de5ux4e4bux5de7}}
\addcontentsline{toc}{section}{化工之巧}

化学工业制造出的千千万万种新产品,使人类的物质生活更加丰富多彩,真是``巧夺天工'',``巧夺造化之工''。在制
造过程中,它的化合和分解方式是用化学方程来描述的,但它是在变化的,因此,伟大革命导师恩格斯明确指出:``表示
物体的分子组合的一切化学方程式,就形式来说是微分方程式。但是这些方程式实际上已经由于其中表示的原子量而积
分起来了。化学所计算的正是量的相互关系为已知的微分。''

为了形象化地说明,例如,某种物质中含有硫,用苯提取硫。苯吸取硫有一定的饱含量,在这个过程中,苯含硫越
多越难再吸取硫,剩下的硫越少越难被苯吸取。这个过程时刻都在变化的,吸收过程速度在不断减慢着。实验本身便是
这个过程的积分过程,它的数学表达形式就是微分方程式及其求解。简单易作的过程我们可以用实验去解决。但对于复
杂、难作的过程,则常常需要用数学手段来加以解决。特别是选取最优过程的工艺,数学手段更成为必不可少的手段。
特别是量子化学的发展,使得化学研究提高到量子力学的阶段,数学手段---微分方程及矩阵图论更是必需的数学工
具。

应用了数学方法还可使化学理论问题得到极大的简化。例如,对于共轮分子的能级计算,在共轮分子增大时十分困
难。应用了分子轨道的图形理论,由图象形来简化计算,取得了十分直观和易行的效果,便是一例。其主要根据是如果
一个行列式中的元素为0的多,那就可以用图论来简化计算。

\hypertarget{ux5730ux7403ux4e4bux53d8}{%
\section*{地球之变}\label{ux5730ux7403ux4e4bux53d8}}
\addcontentsline{toc}{section}{地球之变}

我们所生活的地球处于多变的状态之中,从高层的大气,到中层的海洋,下到地壳,深入地心都在剧烈地运动着,
而这些运动规律的研究也都用到数学。

大气环流,风云雨雪,天天需要研究和预报,使得农民可以安排田间农活,空中交通运输可以安排航程。限风等灾
害性天气的预报,使得海军,渔民和沿海地区能够及早预防,减少损害。而所有这些预报都离不了数学。

``风乍起,吹皱一池春水''。风和水的关系自古便有记述,``无风不起浪''。但是风和浪的具体关系的研究,则是近代才
逐步弄清的,而在风与浪的关系中用到了数学的工具,例如偏微分方程的间断解的问题。

大地每年有上百万次的地震,小的人感觉不到,大的如果发生在人烟稀少的地区,也不成大灾。但是每年也有几次
在人口众多的地区的大震,形成大灾。对地壳运动的研究,对地震的预报,以及将来进一步对地震的控制都离不开数学
工具。

\hypertarget{ux751fux7269ux4e4bux8c1c}{%
\section*{生物之谜}\label{ux751fux7269ux4e4bux8c1c}}
\addcontentsline{toc}{section}{生物之谜}

生物学中有许许多多的数学问题。蜜蜂的蜂房为什么要象如下的形式(图4),一面看是正六角形,另一面也是如此。
但蜂房并不是六棱柱,而它的底部是由三个菱形所拼成的。图5是蜂房的立体图。这个图比较清楚,更具体些,拿一支
六棱柱的铅笔未削之前,铅笔一端形状是ABCDEF正六角形(图6)。通过AC,一刀切下一角,把三角形ABO搬置
AOO处。过AE,CE也如此同样切三刀,所堆成的形状就是图7,而蜂巢就是两排这样的蜂房底部和底部相接而成。

关于这个问题有一段趣史:巴黎科学院院士数学家克尼格,从理论上计算为使消耗材料最少菱形的两个角度应该是
109°26'和70°34'。与实际蜜蜂所做出的仅相差2分。后来苏格兰数学家马克劳林重新计算,发现错了的不是小小的蜜
蜂,而是巴黎科学院的院士因克尼格用的对数表上刚好错了一个字。这十八世纪的难题,1964年我用它来考过高中
生,不少高中生提出了各种各样的证明。

这一问题,我写得篇幅略长些,目的在于引出生物之谜中的数学,另一方面也希望生物学家给我们多提些形态的问
题。蜂房与结晶学联系起来,这是``透视石''的晶体。再回到化工之巧,有多少种晶体可以无穷无尽、无空无隙地填满空
间,这又要用到数学。数学上已证明,只有230种。

还有如膜岛素的研究中,由于复杂的立体模型也用了复杂的数学计算。生物遗传学中的密码问题是研究遗传与变异
这一根本问题的,它的最终解决必然要考虑到数学问题。生物的反应用数学加以描述成为工程控制论中``反馈''的泉源。
神经作用的数学研究为控制论和信息论提供了现实的原型。

\hypertarget{ux65e5ux7528ux4e4bux7e41}{%
\section*{日用之繁}\label{ux65e5ux7528ux4e4bux7e41}}
\addcontentsline{toc}{section}{日用之繁}

日用之繁,的确繁,从何谈起真为难!但也有容易处。日用之繁与亿万人民都有关,只要到群众中去,急工农之所
急,急生产和国防之所急,不但可以知道哪些该搞,而且知道轻重缓急。群众是真正的英雄,遇事和群众商量,不但政
治上有提高,业务上也可以学到书本上所读不到的东西。象我这样自学专攻数学的,也在各行各业师傅的教育下,学到
了不少学科的知识,这是一个大学一个专业中所学不到的东西。

我在日用之繁中搞些工作始于大跃进的1958年,但真正开始是1964年接受毛主席的亲笔指示后。并且使我永远不
会忘记的是在我刚迈出一步写了《统筹方法平话》下到基层试点时,毛主席又为我指出:``不为个人而为人民服务,十分欢
迎''的奋斗目标。后来在敬爱的周总理关怀下又摘了《优选法》。由于各省、市、自治区的领导的关怀,我曾有机会到过
二十个省市、下过数以千计的工矿农村,拜得百万工农老师,形成了以具有实践经验的工人为主体的三结合的小分队。通
过群众性的科学实验活动证明,数学确实大有用场,数学方法用于革新挖潜,能为国家创造巨大的财富。回顾已往,真
有``抱着金饭碗讨饭吃''之感。

由于我们社会主义制度的优越性,在这一方面可能有我们自己的特点,不妨结合我下去后的体会多谈一些。

统筹方法不仅可用于一台机床的维修、一所房屋的修建、一组设备的安装、一项水利工程的施工,更可用于整个
企业管理和大型重点工程的施工会战。大庆新油田开发,万人千台机的统筹,黑龙江省林业战线采、运、用、育的统筹,
山西省大同市口泉车站运煤的统筹,太原铁路局太钢和几个工矿的联合统筹,还有一些省市公社和大队的农业生产统筹
等等,都取得了良好效果。看来统筹方法宜小更宜大。大范围的过细统筹效果越好,油水越大。

为了贯彻统筹兼顾、合理安排的思想,统筹方法仅提供了一个辅助工具。初步看来潜力不小,特别是把方法交给广
大群众,结合具体实际,大家动手搞起来,由小到大、由简到繁,在普及的基础上进一步提高,收效甚大。初步设想可
以概括成十二个字:大统筹,理数据,建系统,策发展,使之发展成一门学科---统筹学,以适应我国的具体情况,体
现我们社会主义社会的特点。统筹的范围越大,得到和用到的数据也越多。我们不仅仅是统计这些数据,而且还要从这
些数据中取出尽可能多的信息来作为指导。因此数据处理提到了日程上来。数据纷繁就要依靠电子计算机。新系统的建
立和旧系统的改建和扩充,都必须在最优状态下运行。更进一步就是策发展,根据今年的情况明年如何发展才更积极又
可靠,使国民经济的发展达到最大可能的高速度。

优选法是采用尽可能少的试验次数,找到最好方案的方法。优选学作为这类方法的数学理论基础,已有初步地系统
研究。实践中,优选法的基本方法,已在大范围内得到推广。目前,我国化工、电子、冶金、机械、轻工、纺织、交通、
建材等等方面都有较广泛地应用。在各级党委的领导下,大摘推广应用优选法的群众运动,各行各业搞,道道工序搞,
短期内就可以应用优选法开展数以万计项目的试验。使原有的工艺水平普遍提高一步。在不添人、不增设备、不加或少
加投资的情况下,就可收到优质、高产、低耗的效果。例如,小型化铁炉,优选炉形尺寸和操作条件,可使焦铁比一般可
达1:18。机械加工优选刀具的几何参数和切削用量,工效可成倍提高。烧油锅炉,优选喷枪参数,可以达到节油不冒黑
烟。小化肥工厂搞优选,既节煤又增产。在大型化工设备上搞优选,提高收率潜力更大。解放牌汽车优选了化油器的合
理尺寸,一辆汽车一年可节油一吨左右,全国现有民用汽车都来推广,一年就可节油六十余万吨。粮米加工优选加工工
艺,一般可提高出米率百分之一、二、三,提高出粉率百分之一。若按全国人数的口粮加工总数计算,一年就等于增产
几亿斤粮食。

最好的生产工艺是客观存在的,优选法不过是提供了认识它的、尽量少做试验、快速达到目的的一种数学方法。

物资的合理调配,农作物的合理分布,水库的合理排灌,电网的合理安排。工业的合理布局,都要用到数学才能完满
解决,求得合理的方案。总之一句话,在具有各种互相制约、互相影响的因素的统一体中,寻求一个最合理《依某一目的,
如最经济,最省人力)的解答便是一个数学问题,这就是``多、快、好、省''原则的具体体现。所用到的数学方法很
多,其中确属适用者我们也准备了一些,但由于林彪、``四人帮''一伙的干扰破坏,没有力量进行深入的工作。今天科技
工作要大干快上。用于``当务之急''的数学研究和应用必将出现一个崭新的局面。

\hypertarget{ux6570ux5b66ux4e4bux53d1ux5c55}{%
\section*{数学之发展}\label{ux6570ux5b66ux4e4bux53d1ux5c55}}
\addcontentsline{toc}{section}{数学之发展}

宇宙之大,粒子之微,火箭之速,化工之巧,地球之变,生物之谜,日用之繁,无处不用数学。其它如爱因斯坦用数
学工具所获得的公式指出了寻找新能源的方向,并且还预示出原子核破裂发生的能量的大小。在天文学上,也是先从计
算上指出海王星的存在,而后发现了海王星。又如高速飞行中,由次音速到超音速时出现了突变,而数学出现了混合型
偏微分方程的研究。还有无线电电子学与计算技术同信息论的关系,自动化与控制技术同常微分方程的关系,神经系统
同空制论的关系,形态发生学与结构稳定性的关系等等,不胜枚举。

数学是一门富有概括性的学问。抽象是它的特色。同是一个方程,弹性力学上是描写振动的,流体力学上却描写了
流体动态,声学家不妨称它是声学方程,电学家也不妨称它为电报方程,而数学家所研究的对象正是这些现象的共性的
一面------双曲型偏微分方程。这个偏微分方程的解答的性质就是这些不同对象的共同性质,数值的解答也将是它所联系
各学科中所要求的数据。

不但如此,这样的共性,一方面可以促成不同分支产生统一理论的可能性,另一方面也可以促成不同现象间的相互
模拟性。例如:声学家可以用相似的电路来研究声学现象,这大大地简化了声学实验的繁重性。这种模拟性的最普遍的
应用便是模拟电子计算机的产生。根据神经细胞有兴奋与抑制两态,电学中有带电与不带电两态,数学中二进位数的0
与1、逻辑中的``是''与``否'',因而有用电子数字计算机来模拟神经系统的尝试,及模拟逻辑思维的初步成果。

我们作如上的说明,并不意味着数学家可以自我陶醉于共性的研究之中。一方面我们得承认,要求数学家深入到研
究对象所联系的一切方面是十分困难的,但是这并不排斥数学家应当深入到他所联系到的为数众多的科学之一或其中的
一部分。这样的深入是完全必要的。这样做既对国民经济建设可以做出应有的贡献,而且就是对数学本身的发展也有莫
大好处。

客观事物的出现一般讲来有两大类现象。一类是必然的现象------或称因果律。一类是大数现象------或称机遇律。表
示必然现象的数学工具一般是方程式,它可以从已知数据推出未知数据来,从已知现象的性质推出未知现象的性质来。
通常出现的有代数方程,微分方程,积分方程,差分方程等等(特别是微分方程)。处理大数现象的数学工具是概率论与
数理统计。通过这样的分析便可以看出大势所趋,各种情况出现的比例规律。

数学的其它分支当然也可以直接与实际问题相联系。例如:数理逻辑与计算机自动化的设计,复变函数论与流体力
学,泛函分析与群表示论与量子力学,黎曼几何与相对论等等。在计算机设计中也用到数论。一般说来,数学本身是一个
互相联系的有机整体、而上面所提到的两方面是与其它科学接触最多最广泛的。

计算数学是一门与数学的开始而俱生的学问,不过今天由于快速大型计算机的出现特别显示出它的重要性。因为对
象日繁,牵涉日广(一个问题的计算工作量大到了前所未有的程度)。解一个一百个未知数的联立方程是今天科学中常见
的(如水坝应力,大地测量,设计吊桥,大型建筑等等),仅靠笔算就很困难。算一个天气方程,希望从今天的天气数据
推出明天的天气数据,单凭笔算要花成年累月的时间。这样算法与明天的天气何干?一个讽刺而已!电子计算机的发明
就满足了这种要求。高速度大存储量的计算机的发展改变了科学研究的面貌,但是近代的电子计算机的出现丝毫没有减
弱数学的重要性,相反地更发挥数学的威力,对数学的要求提得更高。繁重的计算劳动减轻了或解除了,而创造性的劳
动更多了。计算数学是一个桥梁,它把数学的创造同实际结合起来。同时它本身也是一个创造性的学科。例如推动了一
个新学科计算物理学的发展。

除掉上面所特别强调的分支以外,并不是说数学的其余部分就不重要了。只有这些重点部门与其它部分环环扣紧,
把纯数学和应用数学都分工合作地发展起来,才能既符合我国当前的需要,又符合长远需要。

从历史上数学的发展的情况来看。社会愈进步,应用数学的范围也就会愈大,所应用的数学也就愈精密,应用数学
的人也就愈多。在日出而作,日入而息的古代社会里,会数数就可以满足客观的需要了。后来由于要定四时,测田亩,
于是需要窥天测地的几何学。商业发展,计算日繁,便出现了代数学。要描绘动态,研究关系的变化,变化的关系,因
而出现了解析之学、微积分等等。

数学的用处在物理科学上已经经过历史考验而证明。它在生物科学和社会科学上的作用也已经露出苗头。在在着十
分宽广的前途。

最后,我得声明一句,我并不是说其它科学不重要或次重要。应当强调的是,数学之所以重要正是因为其他科学的
重要而重要的,不通过其他学科,数学的力量无法显示,更无重要之可言了。

需要指出的是,``四人帮''为了复辟资本主义,疯狂地破坏生产,破坏科学技术的发展,他们既破坏理论研究工作,
更疯狂地打击从事应用数学的工作者,他们要的是无文化的体力劳动者,供他们剥削。他们的遗毒需要彻底清除,不可
低估。为了在本世纪末实现``四个现代化'',为把我国建成强大的社会主义国家这一伟大目标,发展数学的重要性是无可
置辩的。

\hypertarget{ux8c08ux8c08ux6570ux5b66ux6253ux57faux7840ux95eeux9898}{%
\chapter*{谈谈数学打基础问题}\label{ux8c08ux8c08ux6570ux5b66ux6253ux57faux7840ux95eeux9898}}
\addcontentsline{toc}{chapter}{谈谈数学打基础问题}

学数学,我一向提倡学生多演算一些习题,这有利于弄清基本概念、定义,以至于达到熟练运算的程度。
这就是非常重要的基本功。我青年时代学数学,就养成多解题的习惯。我首先把教科书上的规定的习题,
通过自己独立思考把它解出来,从中领会其所依据的是什么基本概念,什么定理,然后思考是否还能用
别的方法解题,把这道习题真正搞懂。有一次,我曾用大同小异的二十种方法解一道题,互相比较,从
中找出最简便的一种方法。由于基本功扎实了,后来我在演算其他题目是,就感到运用自如。

伹是,目前有人对解题的意义认识有偏差,以为题目做得越多,考试命中率就越高。于是家长、老师忙于收集各类
习题,历届高考试题,竞赛题,外国中学生试题等,统统要求学生做出来,幻想从中碰上升学的试题。这种``题海战术''
式的解题,实际上是一种变相的下赌注。它不利于学生基本功的训练,对提高学习成绩收效甚微。其实,每年高考,数
学试题的类型就那么一些,但题目却变化万千。你做这题,他出那题,完全相同的题几乎没有。如果定理不熟,不会应
用,有时看上去象是``对上号'',结果还是要解错的。所以,``题海战术''只能加重学生的负担,不利于学生的健康成长,
不可能真正提高教学质量。前几年,我收到一位外省青年学生寄来十公斤重的笔记本,打开一看,全是数学习题解答,起
码有七八千道。他附信说,自己解了这么多题,数学高考还是考不好,身体也搞坏了,这件事足以发人深省。

有的青年写信问我学好数学有什么``秘诀''。我想了-下,认为学数学要打好基础,是一个根本问题。初等数学的
基础就是算术、代数、几何、三 角,学好这些学科才有可能学高等数学,高等数学的基础应该说是解析几何和微积分,
由此可见:算术、代数、几何、三 角是数学的基础的基础。中小学生要在解数学习题上能够做到多快好省,必须牢固地
学好这四门学科的基础知识,这应该说是唯一的``秘诀''罢。所谓``学好'',是指把各学科的内容即教科书内容包括其中所
有习题学得深透,演算得烂熟,真正做到没有一个定理不会证,没有一个习题不会做的程度。

我们为什么要演算习题呢?第一,是为了加深对书本中的基本概念、定义和定理的理解,这是主要的。第二,也是
为丁训练我们的运算技巧和逻辑思维。这虽是次要的,但是必不可缺的。习题是不可不做的,多做些习题,对于加深理解
和提高运算技巧,逻辑思维都是有利的。但必须指出,光靠演算习题而忽视学深学透教科书中的基本概念、定义、定理
(包括征明),首先是学不好数学,因而也演不出改头换面的习题来的。

我这里举一个数学竞赛试题为例,来说明学深学透教科书中的基本概念、定义、定理的重要性。题目是这样的:经
过一条定二次曲线上的二定点A和B,任意作圆,使和二次曲线再交于另外二点C和D,求证连线CD常有一定的方
向。这确是什么书上都没见到过的试题,乍看起来是有点难。我们的看法是这样: 大部分考生做不出这道题,不是因为问
题太难太偏、而是因为他们没有学深学透解析几何的基本内容。哪些是解机几何的基本内容呢?

第一,为了讨论几何图形的方便,我们要恰当地选择坐标系(指的是正交坐标系),使问题中有关图形的表达式化为简单
的形式。在上述的试题里,我们选A为坐标原点O,直线AB为x轴,于是B的坐标便为(1,0),而AB的方程化为y=0。

第二,二次曲线的方程关于x,y是二次的。比方说\(f(x, y)=0\),由于要通过A和B,所以方程是:
\[f(x, y)=x^2+2hxy+ by^2-lx+ cy = 0\]
式中h,b,c都是已知的,至于经过A和B的任意元,它也具有同一形式的方程,不过想x\textsuperscript{2和y}2的系数相等面且没有\(xy\),
这是圆方程的特点,所以圆的方程是:
\[g(x,y)=x^2+y^2-lx+my=0\]
式中的\(m\)是任意的。

第三,两条二次曲线一般有四个交点。就上列的二次曲线和圆看,A,B,C,D就是这四个交点。 现在我们作方
程:
\[f(x,y) - g(x,y)=0\]
并研究这条新的二次曲线。可是这个方程可以写成;
\[2hx+(b-1)y+c-m=0\]
所以它表示经过四个交点的两条直线。由于 \(y=0\) 表示直线AB,所以
\[2hx+(b-1)y+c-m=0\]
应该是直线CD的方程了。从此看出直线CD的斜率\(\frac{2h}{1-b}\)是常数(和\(m\)无关)。

我恳切地希望广大中学同学在中学数学教师的指导下,学好各门数学的基础知识,有计划、有步骤地练好做习题的
计算技巧和逻辑思维,适当挑选一些辅助习题而不是盲目贪多地去找习题。我也恳切地希望中学数学教师指导青年人更
好地学习数学基础知识,使他们更快地成长起来。

\hypertarget{ux8981ux8bb2ux7a76ux5b66ux4e60ux65b9ux6cd5}{%
\chapter*{要讲究学习方法}\label{ux8981ux8bb2ux7a76ux5b66ux4e60ux65b9ux6cd5}}
\addcontentsline{toc}{chapter}{要讲究学习方法}

我常收到一些中学生来信,谈及学习中遇到的种种问题。可以看出,这些同学都有一股强烈的求知欲,也想使自
已的成绩出众,然面或绩总是提不上去。他们不服气,因为论努力的程度,并不亚于成绩比自己好的同学。成绩
不佳的原因是多方面的,但从来信中谈到的情况看,不讲究学习方法,不重视打好基础,恐怕也是学习成绩差的原因之一。

常言道,过河需要有桥,学习不能不注意学习方法。一般说来,学习成绩好的同学,大多数都能联系自己的实际情
况,讲出几种行之有效的学习方法,譬如解题时要注意审题,运算当中要防止粗枝大叶,演算要多采用几种方法,得出结
果之后应该验算等等。同学们可以学习别人的好方法,也可以靠自己在学习中开动脑筋,不断摸索和积累新的方法。但我
总觉得,在研究学习方法时,一定不要忘记打好基础和改进学习方法之间的关系。这篇短文不可能细谈具体的某个学习
方法,我只想针对同学中提出的一两个较普遍的问题,谈谈自己的看法。

``我看到题目,自己想不出解题的方法,当别人 稍加提示,就能做出来。''有的同学来信淘问,这究竟是个什么问
题? 依我看,这主要是基础知识掌握得不扎实,对概念和定理没有真正弄懂。我们为什么要演算习题呢?第一,是为了
加深对书本中的基本概念、定义和定理的理解,这是主要的。第二,也是为了训练我们的运算技巧和逻辑思维。这虽是次
要的,但是必不可少的。做习题,对于加深理解和提高运算技巧、逻辑思维都是有利的。但必须指出,光靠演算习题而
忽视学深学透教科书中的基本概念、定义、定理(包括证明),肯定是学不好数学的。所以,我们在解题时,首先要看消楚
这道题包含了哪些基础知识,会用到哪几个公式或定理,然后从某个公式或定理下手,一步步将题解出来。有的同学需
要别人提示才能做出来,这就是自己不知从何处下手,需要别人给予具体的指点,说明他对需要运用的定理或公式没有
真正弄懂,在使用中要么无从下手,要么下了手把题目解错了。由此可见,在学习中我们必须反对不懂装懂的不良学风,
懂就懂,绝对不能使用``不太懂''这类含糊其词的话米对待学习。

``我做习题,单个的定理或公式的运用,还可以,遇到综合性的题目,就怎么也想不出解题的办法。''这毛病又出
现在哪呢? 我认为,这些同学对定理或公式虽然懂了,但这大多是靠死记硬背的,对这些定理之间的内在联系缺乏了
解,更没有达到融会贯通的程度,这应当说是不熟的毛病。定理怎么可以靠死记硬背来掌握呢?我们要学好数学,这个
``学好'',我理解是把算术、代数、几何、三角这几门基础学科的内容,即教科书内容包括其中所有习题学得深透,演
算得烂熟,真正做到没有一个定理不会证,没有一个习题不会做的程度。这样,遇到综合题,就能把几个单一的定理公
式融会贯通起来思考,再加上熟能生巧,综合题就不难解出来了。当然,要达到这样的程度,这并不是显而易到的事。
譬如有的平面几何题目,在运算解题中需要画一条辅助线,这条线画在什么地方大有讲究,这确实是有一定的难度。画
准了,一下子就能解出来,画不准,就可能解不出来,即使解出来,也可能是错误的答案。那么,怎样才能有把握一画
即准呢?这就要靠平时的锻炼,按类型来练,演算多了,积累的经验也就丰富了。从另一个方面来说,演算习题还要经
受失败的考验。有的同学一发现解不出,就不肯多动脑筋,总希望别人指点,这样自己就缺乏独立解题的能力,稍为难
点的题目就望而生畏,缩手缩脚,这也是不利于解题能力的。所以,从某种意义上说,经过自己解出的题目,尽管时
间多花了一点,但对定理和公式的理解和运用,印象就深刻得多了。

``我的脑子笨,也许不是学数学的料子。''有的同学在解题中多次碰到困难之后,往往容易产生悲观情绪和无所作为
的思想。有的同学则显得非常急躁,到处求师,希望能得到一套现成的学习方法,使他在几天之内变聪明起来,学习成
绩一下子提上去。这种想法是不现实的。因为学习这东西,是有其规律性的,必须由浅入深,由易到难,由低到高,循
序渐进,别人的学习方法是他自己总结出来的,是针对他自己的情况而定的。别人的好方法,不能生搬硬套。目前学习
成绩较差的同学不要悲观,不要性急,须知,欲速则不达。正确的态度是,既向同学学习好方法,为我所用,更重要的
是,要下决心,从打基础抓起,一点一滴,扎扎实实,把所学的定理、公式及其证明真正搞懂,弄熟,这样也许时间多
花了,效果也许慢了一点,但提高成绩的效果也许会明显一点。

\hypertarget{ux5b66ux4e60ux6570ux5b66ux7684ux51e0ux70b9ux4f53ux4f1a}{%
\chapter*{学习数学的几点体会}\label{ux5b66ux4e60ux6570ux5b66ux7684ux51e0ux70b9ux4f53ux4f1a}}
\addcontentsline{toc}{chapter}{学习数学的几点体会}

\hypertarget{ux8ba4ux8bc6ux5b66ux4e60ux6570ux5b66ux7684ux91cdux8981ux6027}{%
\section*{认识学习数学的重要性}\label{ux8ba4ux8bc6ux5b66ux4e60ux6570ux5b66ux7684ux91cdux8981ux6027}}
\addcontentsline{toc}{section}{认识学习数学的重要性}

从远古到今天,人们都要和数学打交道。1,2,3,\ldots{}这些简单的整数,从日常生活到尖端科学技术都离不开它们。
由于生产的需要而引起了初等几何学的产生与发展。每一门学问,当它们需要用数量来精确刻划时,都离不开数学。科
学技术越发展,就越需要更多更深的数学作为工具。特别是在电子计算机产生之后,更为数学方法的广泛使用与渗透,开
辟了前所未有的广阔前景。由于计算数量太大,不少过去以为不可能计算的问题,现在都有可能计算了。不仅是力学、
天文学、物理学、工程技术等这些经常需要用数学的领域,即使过去不常用数学的领域,如化学、物理学、经济学与工厂
企业管理等,都日益离不开数学,而且用到很深的数学。值得指出的是,有了好的数学逻辑思维的训练,再去学习与研
究别的很多领域,往往是易于深入的。所以数学几乎是所有人必须掌握的强有力的工具。因此可以说,数学的重要,正
表现在其他学科的重要是它们对于数学的需要上面。

\hypertarget{ux638cux63e1ux57faux672cux6982ux5ff5}{%
\section*{掌握基本概念}\label{ux638cux63e1ux57faux672cux6982ux5ff5}}
\addcontentsline{toc}{section}{掌握基本概念}

从远古到今天,人们都要和数学打交道。1,2,3,\(\cdots\)这些简单的整数,从日常生活到尖端科学技术都离不开它们。
由于生产的需要而引起了初等几何学的产生与发展。每一门学问,当它们需要用数量来精确刻划时,都离不开数学。科
学技术越发展,就越需要更多更深的数学作为工具。特别是在电子计算机产生之后,更为数学方法的广泛使用与渗透,开
辟了前所未有的广阔前景。由于计算数量太大,不少过去以为不可能计算的问题,现在都有可能计算了。不仅是力学、
天文学、物理学、工程技术等这些经常需要用数学的领域,即使过去不常用数学的领域,如化学、物理学、经济学与工厂
企业管理等,都日益离不开数学,而且用到很深的数学。值得指出的是,有了好的数学逻辑思维的训练,再去学习与研
究别的很多领域,往往是易于深入的。所以数学几乎是所有人必须掌握的强有力的工具。因此可以说,数学的重要,正

不断地抽象是数学的特性之一。1,2,3,\ldots{}是一个人,一个桔子,两张桌子等等的抽象,其四则运算可以概括无穷
多个具体事物间的四则运算。初等几何中的直线与圆等,在现实生活中都是没有的。所谓圆就是与一点等距离的点的轨
迹。现实生活中的圆盘子与脸盆的周围只是近似的圆。但许多具体对象,一经抽象,就能概括得更广泛。由此而得到的
规律,可以在更广泛的实践中得到运用。学习数学时,不断会碰到新的抽象概念。在小学时,会碰到整数、质数、分数、
小数的概念。中学的代数,用文字代替具体的数字,除普通平面几何外,在解析几何中,引进了直角坐标系,所以空间
的点有了坐标表示,从而可以用代数方法处理几何问题。进了大学,微积分的第一关就是新加入了无穷小与无穷大的概念,
极限的概念。如此等等。

学习数学首先就要弄清楚一个个概念。否则脑子里难免是一盆浆糊。掌握抽象概念的办法是多想想具体的例子.例如
二分之一等于四分之二,可以说想象为一块月饼的半块等于两块四等分块月饼。中学的代数方程可以看作一些数满足的
关系式,其中有一个未知数。用数的四则运算法则,可以将这个未知数求出来。这样与算术就可以一致起来。有了未知数
的引进,就可以将算术的四则应用题用方程表示出来,十分易解。而这类问题。如果用算术方法去做,则要困难得多。
解析几何也是同样的,将平面几何中的点,直线等都用数字(或文字)与方程表示出来,因此问题远比平面几何方法解决
来得容易。在学到极限时,更要多掌握些具体数列求极限的例子。

所以要牢牢地掌握基本概念,一是要多了解这一概念的具体例子,二是要对照过去学过的东西,看看有了新概念与方法,到底
有什么优越性?这样还可以把学过的东西有机地联系起来,加深对新概念的理解。

\hypertarget{ux52e4ux4e8eux601dux8003ux4e0eux52a8ux624b}{%
\section*{勤于思考与动手}\label{ux52e4ux4e8eux601dux8003ux4e0eux52a8ux624b}}
\addcontentsline{toc}{section}{勤于思考与动手}

仅仅弄清楚概念还不够,还要勤于动手。否则不仅会眼高手低,而且对概念的了解也很浮浅。在学校学习数学的阶
段,适当地多做一些习题是很有必要的,真正弄清楚一个新的概念与方法的标准是看掌握运用它们的水平如何。而这
些是需要通过做习题,即通过独立思考,自己多动手做才能学会。因此一定要勤于思考与动手。当然,同样水平的题目,
特别是代公式的习题、做它许许多多,也是不必要的。要做一些经过思考以后才能够得到答案的较难的习题。更要养成
一个好习惯,就是不要看不起容易的东西,不要放过每一个可以锻炼自己的机会。我国著名数学家华罗庚教授说过,``搞
数学要拳不离手,曲不离口''。他是善于利用一切机会思考数学问题的榜样。例如 1961 年报纸登载苏联向太平洋海域发
射火箭,公报中给出了由四个点构成的一个火箭头着落区城,他就由此去推算出火箭的发射点。又如 1953 年,匈牙利
数学家保尔·吐朗来我国访问时,曾报告了由他给出的我国古代数学家李善兰的一个恒等式的证明,其中用了不少高深
数学知识。华老连夜思考,终于在吐朗离北京前夕,给了李善兰恒等式一个纯初等的证明,并将结果告诉吐朗,为我国
数学家争了气。我跟华老学习与工作,深深体会到这是一个很好的方法。例如,到本世纪末,我国工农业总产值要翻两番。
利用对数知识可以算出年递增率为百分之十七点二左右。这个结论不用对数知识可不可以算出来? 我算过,是可以的。

华老还教导我们,在遇到一个数学问题时,要学会分析,逐步将问题化简,最后加以解决。解决之后,要多想想是不
是走了弯路? 有没有更简单的证明? 要这样反复推敲提高。还要学会心算,学会利用零星时间想问题。他告诉我们在学
习一条定理时,一定要多想想,多分析一下这个定理是怎样创造出来的。这样的分析不仅可以破除对著名数学家的迷
信,对于培养与提高自己以后独立解决问题的能力会有很大的作用。在我跟华老学习与工作的三十多年里,曾多次看到
他解决一些数学问题的全部思索过程。这对培养我学数学与研究数学的能力是受益不尽的。

要勇敢的暴露自己的嵘点。争取别人的帮助。中国民间有旬话:``不要重门弄斧''。说的是你不要到鲁班面前去玩弄
斧头。意思是不要到专家内行面前去谈论他们内行的东西,以免暴露缺点,弄销丢脸。在旧社会,为了生计,这可以说
是一句处事格言。但这决不是鼓励人上进的话。华老把它改为``必需到班门弄斧''。因为只有到班门去弄斧,才能得到最
多的批评与帮助。华老是最善于向别人的优点学习的人。他听别人演讲时,常常打破锅子问到底。有时还对别人的工作
加以简化提高。对学生则更为严格,学生演讲时,常常被问得下不了讲台。他写的东西,在发表前,总是让学生看看,
提的意见愈多,他就愈高兴,师生关系也就越亲密。不少东西,他还专门寄交国内外专家提意见,争取得到帮助。
在他出国讲学时,每到一个学校,专门讲对方特别高明的东西,华老这样做,就是为了从他们那里得到真正的批评与
更多的帮助。华老之所以能够成为有多方面卓越贡献的数学家,除他本人的天分与努力之外,跟他善于学习别人的长处
也是分不开的。还应指出,有些人专喜欢做偏题怪题,所谓偏题怪题都是孤立的东西,只不过是玩弄一些特殊技巧而已,
并无助于加深对数学基本概念与内容的了解。这种学习方法是不可取的。

\hypertarget{ux5faaux5e8fux524dux8fdb}{%
\section*{循序前进}\label{ux5faaux5e8fux524dux8fdb}}
\addcontentsline{toc}{section}{循序前进}

学数学最怕的是吃夹生饭。如果一些东西学得糊里糊涂,再继续往前学,则一定越学越糊涂。结果将是一无所获。
所以不要怕学得慢,一定要学得踏实。常常起步慢一些,只要学得很踏实,后来就会快起来的。学习要少而精,力求达
到彻底掌握。华老常说:``要正确估计自己,如果不行,就要退,一直退到懂的地方再继续前进。''华老问我们问题时,
回答一定要求干脆,懂就说懂。不懂就说不懂。说什么``大致差不多''之类不确切的话、他是决不允许的。学数学更不可
以打肿了脸充胖子,假装懂了很多东西。现在有不少人,连初等数学都不掌握,更不用说对近代的数论文献有所了解。
却听不进好言劝告,在那里``研究''费马大定理和哥德巴赫猜想,浪费了很多宝贵的光阴,毫无所获。反而将很多错误的
东西当成正确的东西,印在脑子里,对以后的学习与工作,都将起着很坏的作用。这当然是一种极端的情况。
但确有不少人,当学到一定水平后,就再也学不进去了。究其原因,就是某些部分吃了夹生饭之故。

当然也要防止永远停留在一个水平上,不再继续前进的偏向。有些人掌握了一些工具,例如初等数学,就停留在这
个水平上,用这些工具解决一些问题,不再继续学习与提高。结果是观点很低,水平很低。这是又一种偏向。特别是一帆
风顺,有了一些成绩的人容易犯这个毛病。应该承认学习与掌握一个自己不熟悉的新东西是很费劲的,而且有时还要冒
点风险。这就是要向自己的懒惰与安于现状的心理作斗争。关于这一点,我们特别要向华老学习,他永远不知满足。当
他在四十年代,已经成为国际著名数论学家后,不停步。宁肯另起炉灶,换抽象代数学与复分析。在他老年时,还从事
应用数学的研究与普及。他是既很踏实,又不保守,在数学上,做到了活到老,学到老,工作到老。他的治学经验是值
得我们很好地研究,继承与学习的。

以上是几点不成熟的意见,不妥之处,还望读者不各指教。
	\backmatter
	{\color{TEXTColor}
\end{document}
