\documentclass[twoside,openright,headings=optiontohead]{ctexbook} %{scrbook} %
\renewcommand{\baselinestretch}{1.3}  %行間距倍率
\columnsep 7mm
%\renewcommand\thepage{}
\usepackage{setspace}


\usepackage[
b5paper=true,
%CJKbookmarks,
unicode=true,
bookmarksnumbered,
bookmarksopen,
hyperfigures=true,
hyperindex=true,
pdfpagelayout = SinglePage,
%pdfpagelayout = TwoPageRight,
pdfpagelabels = true,
pdfstartview = FitV,
colorlinks,
pdfborder=001,
linkcolor=black,
anchorcolor=black,
citecolor=black,
pdftitle={Nothing to Envy},
pdfauthor={Barbara Demick},
pdfsubject={爸爸三定律},
pdfkeywords={爸爸},
pdfcreator={lzf}
]{hyperref}


\usepackage{graphics,graphicx,pdfpages}
\usepackage{caption} %用于取消标题编号 \caption*{abc}



\usepackage{xeCJK}
\providecommand{\tightlist}{%
   \setlength{\itemsep}{0pt}\setlength{\parskip}{0pt}}
   
\usepackage{indentfirst}
\setlength{\parindent}{2.0em}

%正文字体
\setCJKmainfont[Path=fonts/,
BoldFont={fzfsjt.ttf},
ItalicFont={fzssjt.ttf}, %方正书宋简体
BoldItalicFont={fzfsjt.ttf},
SlantedFont={fzfsjt.ttf},
BoldSlantedFont={fzfsjt.ttf},
SmallCapsFont={fzfsjt.ttf}
]{fzfsjt.ttf}
\setCJKsansfont[Path=fonts/]{fzfsjt.ttf}
\setCJKmonofont[Path=fonts/]{fzfsjt.ttf}
\setmainfont[Path=fonts/]{fzfsjt.ttf}
\setsansfont[Path=fonts/]{fzfsjt.ttf}
\setmonofont[Path=fonts/]{fzfsjt.ttf}
% Icon 字体
\newfontfamily{\FA}[Path=fonts/]{FontAwesome.otf} %License: SIL OFL 1.1
\newfontfamily{\EA}[Path=fonts/]{EyesAsia-Regular.otf} %License: MIT
\newfontfamily{\EN}[Path=fonts/]{SourceSansPro-ExtraLightIt.otf} %License: SIL OFL 1.1

% 頁面及文字顏色
\usepackage{xcolor}
\definecolor{TEXTColor}{RGB}{50,50,50} % TEXT Color
\definecolor{PinYinColor}{RGB}{180,180,180} % TEXT Color
\definecolor{NOTEXTColor}{RGB}{0,0,0} % No TEXT Color
\definecolor{BGColor}{RGB}{240,240,240} % BG Color
\definecolor{Gray}{RGB}{246,246,246}


\usepackage{multicol}

\makeindex
\renewcommand{\contentsname}{{数学家谈怎样学数学}}
% \usepackage{fancyhdr} % 設置頁眉頁腳
% \pagestyle{fancy}
% \addtolength{\headwidth}{\marginparsep}
% %\addtolength{\headwidth}{\marginparwidth}
% \fancyhf{} % 清空當前設置
% \renewcommand{\headrulewidth}{0pt}  %頁眉線寬,設為0可以去頁眉線
% \renewcommand{\footrulewidth}{0pt}  %頁眉線寬,設為0可以去頁眉線


\usepackage{titleps}% http://ctan.org/pkg/{titleps,lipsum}
\newpagestyle{newstyle}{
  \setheadrule{.4pt}% Header rule
  \sethead[\scriptsize {\FA \ }数学家谈怎样学数学]% even left
    []% even centre
    [{\tiny{\textcolor{Gray}{\FA \ }}}\thepage]% even right
    {{\tiny{\textcolor{Gray}{\FA \ }}}\thepage}% odd left
    {}% odd centre
    {\scriptsize {\FA \ }数学家谈怎样学数学}% odd right
}

\pagestyle{newstyle}



\usepackage{titletoc}
\dottedcontents{section}[100em]{\bfseries}{100em}{100em} % 去掉目录虚线

\usepackage{geometry}
\geometry{b5paper, left=2cm,right=2cm,top=2.5cm,bottom=2cm,foot=2.5cm}
%\usepackage[b5paper,tmargin=2.5cm,bmargin=2.5cm,lmargin=3.5cm,rmargin=2.5cm]{geometry}
\newcommand{\Icon}{\fontsize{600pt}{\baselineskip}\selectfont}

\begin{document}
	\frontmatter
%	\begin{figure}[ht]
%		\begin{center}
%			\includepdf[height=\paperheight]{images/cover.jpg}
%		\end{center}
%	\end{figure}
\newpage
{\color{TEXTColor}
	%\begin{multicols}{2}
		\tableofcontents
	%\end{multicols}
	\newpage
	\mainmatter
% \fancyhead[LO]{{\scriptsize {\FA \ } 数学家谈怎样学数学}}%奇數頁眉的左邊
% %\renewcommand{\headrulewidth}{0pt} % optional
% %\fancyhead[L]{\nouppercase{\leftmark}}
% \fancyhead[RO]{{\tiny{\textcolor{Gray}{\FA \ }}}\thepage}
% \fancyhead[LE]{{\tiny{\textcolor{Gray}{\FA \ }}}\thepage}
% \fancyhead[RE]{{\scriptsize {\FA \ }\leftmark}}%偶數頁眉的右邊
% \fancyfoot[LE,RO]{}
% \fancyfoot[LO,CE]{}
% \fancyfoot[CO,RE]{}

\mainmatter

\hypertarget{ux5b66ux4e60ux548cux7814ux7a76ux6570ux5b66ux7684ux4e00ux4e9bux4f53ux4f1a}{%
\chapter*{学习和研究数学的一些体会}\label{ux5b66ux4e60ux548cux7814ux7a76ux6570ux5b66ux7684ux4e00ux4e9bux4f53ux4f1a}}
\addcontentsline{toc}{chapter}{学习和研究数学的一些体会}

人贵有自知之明。我知道,我对科学研究的了解不全面。也知道,搞科学极重要的是独立思考,各人应依照各人自己的特点找出最适合的道路。听了别人的学习、研究方法,就以为我也会学习研究了,这个就无异于吃颗金丹就会成仙,而无需经过勤修苦练了。

今天把我五十年来的经验教训,所见所闻、所体会的给你们介绍,目的在于尽可能把我的经验作为你们的借鉴,具体问题具体分析,具体的个人应当想出最适于自己的有效方法来。

\textbf{我第一点准备和同志们谈的问题是速度、是效率}。速度是实现社会主义现代化的保证。例如说象我这样又老又拐的人,我在前头走你们赶我不费劲,一赶就赶上,而我要赶你们,除非你们躺下来睡大觉,否则我无论如何是赶不上的。现在世界上科学发展很快,我们如果没有超过美国的速度和效率就不可能赶上美国。我们没有超过日本的速度和效率,我们就不可能赶上日本。如果我们的速度仅仅和美、日等国一样,那么也只能是等时差的赶,超就是一句空话。所以说,我们应当首先在速度和效率上超过他们。

但要我们的速度和效率超过他们有没有可能呢?这似乎是一个大问题,其实不然,我在美国呆过,在英国呆过,也在苏联呆过。我看到他们的速度不是神话般的快不可及。我们是赶得上超得过的!我们许多美籍华人,如果他们的速度不能超过一般的美国人的话,也就不会成为现代著名的科学家了。所以事实证明,只要我们努力下功夫,赶超是完全可以的。就以我自己来说,我是三六年到英国的,在那里呆了两年,回国后在昆明乡下住了两年,四零年就完成了堆垒素数论的工作。后来五零年回国后,在五八年之前,我们的数论、代数、多复变函数论等等都达到了世界上的良好的水平。所以经验告诉我们纯数学的一门学科,有四五年就能在世界上见头角了。你们现在时代更好了,党中央一举粉碎了``四人帮'',带来了科学的春天。在这样的条件下边,我敢断言,只要肯下功夫,努力钻研,只要不浪费一分一秒的时间,我们是能够赶上世界先进水平的。特别是我们数学,前有熊庆来、陈建功、苏步青等老前辈的榜样,现在又有许多后起之秀,更多的后起之秀也一定会接踵而来。\\
\textbf{二、消化} 抢速度不是越级乱跳,不是一本书没有消化好就又看一本,一个专业没有爬到高处就又另爬一个山峰。我们学习必须先从踏踏实实地读书讲起,古时候总说这个人``博闻强记''``学富五车''。实际上古人的这许多话到现在已是不足为训了。五车的书,从前是那种大字的书,我想一个指甲大小的集成电路就可以装它五本十本,学书五车,也不过十几块几十块集成电路而已。现在也有相似的看法,说某人念了多少多少书,某人对世界上的文献记的多熟多熟,当然这不是不必要的,而这只能说走了开始的第一步,如果不经过消化,实际抵不上一个图书馆,抵不上一个电子计算机的记忆系统。人之所以可贵就在于会创造,在于善于吸收过去的文献的精华,能够经过消化创造前人所没有的东西。不然人云亦云世界就没有发展了,懒汉思想是科学的敌人,当然也是社会发展的敌人。

什么是消化?检验消化的最好的方法就是``用''。会用不会用,不是说空话,而是在实际中考验。碰到这个问题束手无策,碰到那个问题又是一筹莫展,即使他能写几篇模仿性的文章,写几本抄抄译译的稿著,这同社会的发展又有什么关系呢?当然我不排斥初学的人写几篇模仿性的文章,但绝不能局限于此,须发皆白还是如此。

消化,只有消化后,我们才会灵活运用。如果社会主义建设需要我们,我们就会为社会主义建设服务,解决问题,贡献力量。客观的问题上面不会贴上标签的,告诉你这需要用数论,那个是要用泛函,而社会主义建设所提出来的问题是各种各样无穷无尽的,想用一个方法套上所有的实际问题,那就是形而上学的做法。只有经过独立思考和认真消化的学者,才能因时因地根据不同的问题,运用不同的方法真正解决问题。

当然,刚才说消化不消化只有在实际中进行检验。但是同学们不一定就有那么多的实践机会,在校学习的时候有没有检查我们消化了没有的方法呢?我以前讲过,学习有一个由薄到厚,再由厚到薄的过程。你初学一本书,加上许多注解,又看了许多参考书,于是书就由薄变厚了。自己以为这就是懂了,那是自欺欺人,实际上这还不能算懂。而真正懂,还有一个由厚到薄的过程。也就是全书经过分析,扬弃枝节,抓住重点,甚至于来龙去脉都一目了然了,在没有这条定理前,人家是怎样想出来的,这样才能说是开始懂了,这也是一个检验自己是否消化了的方法。当然,这个方法不如前面那种更踏实。总的一句话,检验我们消化没有,弄通没有的最后标准是实践,是能否灵活运用解决问题。也许有人会说这样念书太慢了。我的体会不是慢了,而是快了。因为我们消化了我们以前念过的书,再看另一本书时,我们脑子里的记忆系统就会排除那些过去弄懂了的东西。而只注意新书中自己还没有碰到过的新东西。所以说,这样脚踏实地的上去,不是慢了而是快了。不然的话囫囵吞枣的学了一阵,忘掉一阵,再学再忘,白费时光事小,使自己``于国于家无望''事大。更可怕的是好高骛远。例如中学数学没学懂,他已读到大学三、四年级的课程,遇到困难,但又不屑于回去复习,再去弄通中学的东西,这样前进,就愈进愈糊涂,陷入泥坑,难于自拔。有时候阅读同一水平的书,如果我们以往的书弄懂了消化了,那么在同一水平书里找找以往书上没有的东西就可以过去了。找不到很快送上书架,找到一点两点就只要把这一两点弄懂就得了,这样读书就快了,不是慢了。

读书得法了,然后看文献,实际上看文献和看书没有什么不同,也是要消化。不过书上是比较成熟的东西,去粗取精,则精多粗少。而文献是刚出来的,往往精少而粗多。当然不排除有些文章,一出来就变成经典著作的情况。但这毕竟是少数的少数。不过多数文章通过不多时间就被人们遗忘了。有了吸取文献的基础,就可以搞研究工作。

这里我还要强调一下独立思考。独立思考是搞科学研究的根本,在历史上,重大的发明没有一个是不通过独立思考就能搞出来你的。当然这并不等于说不接受前人的成就而``独立''``思考''。例如有许多人,搞哥德巴赫猜想,对前人的工作一无所知,这样搞,成功的可能性是很小的。独立思考也并不是说不要攻书,不要看文献,不要听老师的讲述了。书本、文献、老师都是要的,但如果拘泥于这些,就会失去创造力,使学生变成教师的一部分,这样就会愈缩愈小,数学上出了收敛的现象。只有独立思考才能够跳出这个框框,创造出新的方法,创造出新的领域,推动科学的进步。独立思考不是说一个人独自在那里冥思苦想,不和他人交流。独立思考也要借助别人的结果,也要依靠群众和集体的智慧。独立思考也可以补救我们现在导师不够,导师经验较差,导师太忙顾不过来,但这都需独立思考来补救。甚至于象我们过去在昆明被封锁的时候,外国杂志没处来,我们还是独立思考,想出新的东西来,而想出来的东西和外国人并重复。即使有,也别怕。例如说,我青年时在家里发表过几篇文章,而退稿的很多,原因是别人说你的这篇文章那本书里已有此定理了,那篇文章在某书里也已有证明了等等。面对这种情况是继续干呢?还是就泄气呢?觉得上不起学,老是白费时间搞前人所搞过的东西。当时,我并没有这样想,在收到退稿时反而高兴,这是我明白,原来某大科学家搞过的东西,我在小店里也能搞出来。因此我还是加倍继续坚持搞下去了。我这里并不是说过去的文献不要看,而是说即使重复了人家的工作也不要泄气,要对比一下自己搞出来的同已有的有什么区别,是不是他们的比我们的好,这样就学习人家长处,就有进步,如果我们还有长处就增加了信心。

我们有了独立思考,没有导师或文献不全,就都不会成为我们的阻力。相反,有导师我们也还要考虑考虑讲的话对不对,文献是否完整了\ldots{}\ldots{}。总之,科学事业是善于独立思考的人所创造出来的,而不是象我前面所说的等于几块集成电路的那种人创造出来的,因为这种人没有创造性。毛主席指出:研究问题,要由此及彼,由表及里,去粗取精,去伪存真。做到这四点,就非靠独立思考不可,不独立思考就只得其表,取其粗,只能够伪善杂存,无法明辨是非。

三、搞研究工作的几种境界

1.照葫芦画瓢的模仿。模仿性的工作,实际上就等于做一个习题。当然,做习题是必要的,但是一辈子做习题而无新创又有什么意思呢?

2.利用成法解决几个新问题。这个比前面就进了一步,但是我们在这个问题上也应区别一下。直接利用成法也和做习题差不多,而利用成法,又通过一些修改,这就走上搞科学研究的道路了。

3.创造方法,解决问题。这就更进了一步。创造方法是一个重要的转折,是自己能力的提高的重要表现。
4.开辟方向。这就更高了,开辟了一个方向,可以让后人做上几十年,成百年。这对科学的发展来讲就是有贡献。我是粗略地分为以上这四种,实际上数学还有许多特殊性的问题。象著名问题你怎样改进它,怎样解决它,这在数学方面一般也是受到称赞的。在二十世纪初希尔伯特提出了二十三个问题。这许多问题,有些是会对数学的本质产生巨大的影响。费尔马问题我想这是大家都知道的。这个问题如用初等数论方法解决了,那没有发展前途,当然,这样他可以获得``十万马克''。但对数学的发展是没有多大意义的。而Kummer虽没有解决费尔马问题,但他为研究费尔马却创造了理想数,开辟了方向。现在无论在代数、几何、分析等方面,都用上了这个概念,所以它的贡献远比解决一个费尔马问题大。所以我觉得,这种贡献就超过了解决个别难题。

我对同志们提一个建议,取法乎上得其中,取法乎中得其下。研究工作还有一条值得注意的,要攻得进去,还要打得出来。攻进去需要理论,真正深入到所搞专题的核心需要理论,这是人所共知的。可是要打得出来,并不比钻进去容易。世界上有不少数学家攻是攻进去了·但是进了死胡同就出不来了·这种情况往往使其局限在一个小问题里,而失去了整个时间。这种研究也许可以自娱,而对科学的发展和社会主义的建设是不会有作用的。

\textbf{四、我还想跟同学们讲一个字,``漫''}

我们从一个分支转到另一个分攴是把原来所搞分支丢掉跳到另一分支吗?如果这样就会丢掉原来的。而``漫''就是在你搞熟弄通的分支附近,扩大眼界,在这个过程中逐渐转到另一分支,这样,原来的知识在新的领域就能有用,选择的范围就会越来越大。我赞成有些同志钻一个问题钻许多年搞出成果,我也赞成取得成果后用漫的方法逐步转到其它领域。

鉴别一个学问家或个人,一定要同广,同深联系起来看。单是深,固然能成为一个不坏的专家,但对推动整个科学的发展所起的作用,是微不足道的。单是广,这儿懂一点·那儿懂一点·这只能欺欺外行,表现表现他自已博学多才,而对人民不可能做出实质性的成果来。

数学各个分支之间,数学与其它学科之闻实际上没有不可邀越的鸿沟。以往我们看到过细分割,各搞一行的现象,结果呢?哪行也没搞好。所以在钻研一科的同时,把与自己学科或分支相近的书和文献浏览浏览·也是大有好处的。

\textbf{五、我再讲一个``严''字}

不单是搞科学研究需要严,就是练兵也都要从难,从严。至于说相互之间说好听的话,听了谁都高兴。在三国的时候就有两个人,一个叫孔融,一个叫祢衡,祢衡吹捧孔融是仲尼复生。孔融吹捧祢衡是颜回再世。他们虽然相互捧的上了九宵云外,而实际上却是两个饭桶,其下场都被曹操直接或间接地杀死了。当然,听好话很高兴而说好话的人也有他的理论,说我是在鼓励年青人。但是这样的鼓励,有的时候不仅不能把年青人鼓励上去,反而会使年青人自高自大,不再上进。特别是若干年来,我知道有许多对学生要求从严的教师受到冲击。而一些分数给的宽,所谓关系搞得好的,结果反而得到一些学生的欢迎。这种风气只会拉社会主义的后腿。到现在我们要一个老师对我们要求严格些,而老师都不敢真正对大家严格要求。所以我希望同学们主动要求老师严格要求自己,对不肯严格要求的老师,我们要给他们做一定的思想工作,解除他们的顾虑。同样一张嘴,说几句好听的话同说几句严格要求的话,实在是一样的,而且说说好听话大家都欢迎,这有何不好呢?并且还有许多人认为这样是团结好的表现。若一听到批评,就认为不团结了,需要给他们做思想工作了等等。实际上这是多余的,师生之间的严格要求,只会加强团结,即使有一时想不开的地方,在长远的学习、研究过程中,学生是会感到严师的好处的。同时对自己的要求也要严格。大庆三老四严的作风,我们应随时随地、人前人后地执行。

我上面谈到过的消化,就是严字的体现,就是自我严格要求的体现。一本书马马虎虎的念,这在学校里还可以对付,但是就这样毕了业,将来在工作中间要用起来就不行了。我对严还有一个教训,在1964年,我刚走向实践想搞一点东西的时候,在乌蒙磅礴走泥丸的地方,有一位工程师,出于珍惜国家财产的心情,就对我说:``雷管现在成品率很低,你能不能降低一些标准,使多一些的雷管验收下来。''我当时认为这个事情好办。我只要略略降低一些标准,验收率就上去了。但后来在梅花山受到了十分深刻的教训。使我认识到,降低标准1\%,实际就等于要牺牲我们四位可爱的战士的生命。这是我们后来搞优选法的起点。因为已经造成了的产品,质量不好,我们把住关,把废品卡住,但并不能消除由于废品多而造成的损失。如果产品质量提高了,废品少了,那么给国家造成的损失也就自然而然地小了。我这并不是说质量评估不重要,我在1969年就提倡过。不过我们搞优选法的重点就在这里。这就和治病、防病一样,以防为主。搞优选法就是防止次品出现。而治就是出了废品进行返工,但这往往无法返工,成为不治之症。老实说,以往我对学生的要求是习题上数据错一点没有管,但是自从那次血的教训,使我得到深刻的教育。我们在办公室里错一个1\%好像不要紧,可是拿到生产、建设的实践中去,就会造成极大的损失。所以总的一句话,包括我在内,对严格要求我们的人,应该是感谢不尽的。对给我们戴高帽子的人,我也感谢他,不过他这顶帽子我还是退还回去,请他自己戴上。同学们,求学如逆水行舟,不进则退。只要哪一天不严格要求自己,就会出问题。当然,数学工作者,从来没有不算错过题的。我可以这样说一句,天下只有哑巴没有说过错话;天下只有白痴没有错过问题;天下没有数学家没算错过题的。错误是难免要发生的,但不能因此而降低我们的要求,我们要求是没有错误,但既然出现了错误,就应该引以为教训。不负责任的吹嘘,虽然可能会使你高兴,但我们要善于分析,对这种好说恭维话的人要敬而远之。不负责任地恭维人,是旧社会遗留下来的恶习,我们要尽快地把它洗刷掉。当然,别人说我们好话,我们不能顶回去,但我们的头脑要冷静、要清醒,要认识到这是顶一文钱不值的高帽子,对我的进步毫无益处。

实事求是,是科学的根本。如果搞科学的人不实事求是,那就搞不了科学,或就不适于搞科学。党一再提倡实事求是的作风,不实事求是地说话、办事的人,就背离了党的要求。科学是来不得半点虚假的。我们要正确估价好的东西,就是一时得不到表扬.也不要灰心,因为实践会证明是好的。而不太好的东西,就是一时得到大吹大擂,不会多久也就会烟消云散了。我们要有毅力,要善于坚持。但是在发现是死胡同的时候,我们也得善于转移,不过发现死胡同是不容易的,不下功夫是不会发现的。就是退出死胡同时,也得搞清楚它死在何处,经过若干年后,发现难点解决了,死处复活了,我就又可以打进去。失败是经常的事,成功是偶然的。所有发表出的成果,都是成功的经验,同志们都看到了,而同志们哪里知道,这是总结了无数失败的经验教训才换来的。跟老师学习就有这样一个好处,好老师可以指导我们减少失败的机会,更快吸收成功的经验,在这个基础上又创造出更好的东西。还可以看到他的失败的经验,和山穷水尽疑无路柳暗花明又一村地从失败又怎样转到成功的经验,切不可有不愿下苦功侥幸成功的想法。天才,实际上在他很漂亮地解决问题之前是有一个无数次失败的艰难过程。所以同学们千万别怕失败,千万别以为我写了一百张纸了,但还是失败了,我搞一个问题已两年了,而还没有结果等就丧失信心,我们应总结经验,找到我们失败的原因,不再重复我们失败的道路。总的一句话,失败是成功之母。

似懂非懂,不懂装懂比不懂还坏。这种人在科学研究上是无前途的,在科学管理上是瞎指挥的。如果自己真的知己和承认不懂,则容易听取群众的意见,分析群众的意见,尊重专家的意见,然后和大家一起做出决定来,\ldots{}..。特别对你们年青人,没有经过战火的考验(战火的考验是最好的考验,错误的判断就打败仗,甚至于被敌人消灭),也没有深入钻研的经验,就不知道旁人的甘苦。如果没有组织群众性的搞科学研究的锻炼和能力,就必然陷入瞎指挥的陷阱。虽然他(或她)有雄心想办好科学,实际上会造成拆台的后果。所以我要求你们年青人有两条:1、有对科学钻深钻懂一行两行的锻炼。2、能有搞科学实验运动,组织群众,发动群众,把科学知识普及给群众的本领。不然,对四个现代化来说就会起拉后腿的作用。对个人来说一事无成,而两鬓已斑。

当前在两条不可得兼的时候,择其一也可,总之没有农民不下田就有大丰收的事情,没有不在机器边能生产出产品的工人。脑力劳动也是如此,养得肠肥脑满,清清闲闲,饱食终日无所用心的科学家或科学工作组织者是没有的。单凭天才的科学家也是没有的,只有勤奋,才能勤能补拙,才能把天才真正发挥出米。天资差的通过勤奋努力,就可以赶上和超过有天才而不努力的人。古人说,人一能之己十之,人十能之己百之,这是大有参考价值的名言。

\textbf{六、要善于暴露自己}

不懂装懂好不好?不好!因为不懂装懂就永远不会懂。要敢于把自己的缺点和不懂的地方暴露出来,不要怕难为情。暴露出来顶多受老师的几句责备,说你``连这个也不懂'',但是受了责备后不就懂了吗?可是不想受责备,不懂装懂,这就一辈子也不懂。科学是实事求是的学问,越是有学问的人,就越是敢暴露自己,说自己这点不清楚,不清楚经过讨论就清楚了。在大的方面,百家争鸣也就是如此,每家都敢于暴露自己的想法,每家都敢批评别人的想法,每家都接受别人的优点和长处,科学就可以达到繁荣、昌盛。``''四人帮''搞得大家对问題表态不好,不表态也不好,明知不对也不敢暴露,这样就自然产生僵化,僵化是科学的死敌,科学就不能发展。不怕低,就怕不知底。能暴露出来,让老师知道你的底在哪里,就可以因才施教。同时,懂也不要装着不懂。老师知道你懂了很多东西,就可以更快地带着你前进。也就是一句话,懂就说懂,不懂就说不懂,会就说会,不会就说不会,这是科学的态度。

好表现。这似乎是一个坏事,实际也该分析一下,如果自己不了解,或半知半解而就卖弄他的渊博,一这是真正的好表现,这不好。而把自己懂的东西交流给旁人,使别人以更短的时间来掌握我们的长处,这种表现是我们欢迎的,这不是好表现,这是好表现。科学有赖于相互接触,互相交流彼此的长处,这样我们就可以兴旺发达。

我上面所讲的有片面性,更重要的是为人民服务的回题。大家政治理论学习比我好,同时我们这里也没有时间了,就不在这里多讲了。我用一句话结束我的发言。\\
不为个人,而为人民服务。

当然我这篇讲话就是这个主题,但没能充分发挥,不过人贵有自知之明,我对这方面的认识更弱于我对数学的认识了,而政治干部比我搞业务的人就知道的更多,我也就不想在这里超出我的范围多说了。

\hypertarget{ux5e8fux8a00}{%
\chapter*{序言}\label{ux5e8fux8a00}}
\addcontentsline{toc}{chapter}{序言}

那猴在山中,却会行走跳跃,食草木,饮涧泉,采山花,觅树果;与狼虫为伴,
虎豹为群,獐鹿为友,猕猿为亲;夜宿石崖之下,朝游峰洞之中。真是``山中无甲
子,寒尽不知年。''

众猴都道:``这股水不知是那里的水。我们今日赶闲无事,顺涧边往上溜头寻看源流,耍子去耶!''喊一声,都拖男挈女,唤弟呼兄,一齐跑来,
顺涧爬山,直至源流之处,乃是一股瀑布飞泉。但见那:

一派白虹起,千寻雪浪飞。

海风吹不断,江月照还依。

冷气分青嶂,余流润翠微。

潺名瀑布,真似挂帘帷。

众猴拍手称扬道:``好水,好水!原来此处远通山脚之下,直接大海之波。''又道:``那
一个有本事的,钻进去寻个源头出来,不伤身体者,我等即拜他为王。''连呼了三声,
忽见丛杂中跳出一个石猴,应声高叫道:``我进去,我进去!''好猴!也是他:

今日芳名显,时来大运通。

有缘居此地,天遣入仙宫。
	\backmatter
	{\color{TEXTColor}
\end{document}
