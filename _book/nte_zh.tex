\documentclass[twoside,openright,headings=optiontohead]{ctexbook} %{scrbook} %
\renewcommand{\baselinestretch}{1.3}  %行間距倍率
\columnsep 7mm
%\renewcommand\thepage{}
\usepackage{setspace}


\usepackage[
b5paper=true,
%CJKbookmarks,
unicode=true,
bookmarksnumbered,
bookmarksopen,
hyperfigures=true,
hyperindex=true,
pdfpagelayout = SinglePage,
%pdfpagelayout = TwoPageRight,
pdfpagelabels = true,
pdfstartview = FitV,
colorlinks,
pdfborder=001,
linkcolor=black,
anchorcolor=black,
citecolor=black,
pdftitle={Nothing to Envy},
pdfauthor={Barbara Demick},
pdfsubject={爸爸三定律},
pdfkeywords={爸爸},
pdfcreator={lzf}
]{hyperref}


\usepackage{graphics,graphicx,pdfpages}
\usepackage{caption} %用于取消标题编号 \caption*{abc}



\usepackage{xeCJK}
\providecommand{\tightlist}{%
   \setlength{\itemsep}{0pt}\setlength{\parskip}{0pt}}
   
\usepackage{indentfirst}
\setlength{\parindent}{2.0em}

%正文字体
\setCJKmainfont[Path=fonts/,
BoldFont={fzfsjt.ttf},
ItalicFont={fzssjt.ttf}, %方正书宋简体
BoldItalicFont={fzfsjt.ttf},
SlantedFont={fzfsjt.ttf},
BoldSlantedFont={fzfsjt.ttf},
SmallCapsFont={fzfsjt.ttf}
]{fzfsjt.ttf}
\setCJKsansfont[Path=fonts/]{fzfsjt.ttf}
\setCJKmonofont[Path=fonts/]{fzfsjt.ttf}
\setmainfont[Path=fonts/]{fzfsjt.ttf}
\setsansfont[Path=fonts/]{fzfsjt.ttf}
\setmonofont[Path=fonts/]{fzfsjt.ttf}
% Icon 字体
\newfontfamily{\FA}[Path=fonts/]{FontAwesome.otf} %License: SIL OFL 1.1
\newfontfamily{\EA}[Path=fonts/]{EyesAsia-Regular.otf} %License: MIT
\newfontfamily{\EN}[Path=fonts/]{SourceSansPro-ExtraLightIt.otf} %License: SIL OFL 1.1

% 頁面及文字顏色
\usepackage{xcolor}
\definecolor{TEXTColor}{RGB}{50,50,50} % TEXT Color
\definecolor{PinYinColor}{RGB}{180,180,180} % TEXT Color
\definecolor{NOTEXTColor}{RGB}{0,0,0} % No TEXT Color
\definecolor{BGColor}{RGB}{240,240,240} % BG Color
\definecolor{Gray}{RGB}{246,246,246}


\usepackage{multicol}

\makeindex
\renewcommand{\contentsname}{{title}}
% \usepackage{fancyhdr} % 設置頁眉頁腳
% \pagestyle{fancy}
% \addtolength{\headwidth}{\marginparsep}
% %\addtolength{\headwidth}{\marginparwidth}
% \fancyhf{} % 清空當前設置
% \renewcommand{\headrulewidth}{0pt}  %頁眉線寬,設為0可以去頁眉線
% \renewcommand{\footrulewidth}{0pt}  %頁眉線寬,設為0可以去頁眉線


\usepackage{titleps}% http://ctan.org/pkg/{titleps,lipsum}
\newpagestyle{newstyle}{
  \setheadrule{.4pt}% Header rule
  \sethead[\scriptsize {\FA \ }title]% even left
    []% even centre
    [{\tiny{\textcolor{Gray}{\FA \ }}}\thepage]% even right
    {{\tiny{\textcolor{Gray}{\FA \ }}}\thepage}% odd left
    {}% odd centre
    {\scriptsize {\FA \ }title}% odd right
}

\pagestyle{newstyle}



\usepackage{titletoc}
\dottedcontents{section}[100em]{\bfseries}{100em}{100em} % 去掉目录虚线

\usepackage{geometry}
\geometry{b5paper, left=3cm,right=3cm,top=4cm,bottom=3cm,foot=4cm}
%\usepackage[b5paper,tmargin=2.5cm,bmargin=2.5cm,lmargin=3.5cm,rmargin=2.5cm]{geometry}
\newcommand{\Icon}{\fontsize{600pt}{\baselineskip}\selectfont}

\begin{document}
	\frontmatter
%	\begin{figure}[ht]
%		\begin{center}
%			\includepdf[height=\paperheight]{images/cover.jpg}
%		\end{center}
%	\end{figure}
\newpage
{\color{TEXTColor}
	\begin{multicols}{2}
		\tableofcontents
	\end{multicols}
	\newpage
	\mainmatter
% \fancyhead[LO]{{\scriptsize {\FA \ } title}}%奇數頁眉的左邊
% %\renewcommand{\headrulewidth}{0pt} % optional
% %\fancyhead[L]{\nouppercase{\leftmark}}
% \fancyhead[RO]{{\tiny{\textcolor{Gray}{\FA \ }}}\thepage}
% \fancyhead[LE]{{\tiny{\textcolor{Gray}{\FA \ }}}\thepage}
% \fancyhead[RE]{{\scriptsize {\FA \ }\leftmark}}%偶數頁眉的右邊
% \fancyfoot[LE,RO]{}
% \fancyfoot[LO,CE]{}
% \fancyfoot[CO,RE]{}

\mainmatter

\hypertarget{ux52d8ux8befux8868}{%
\chapter*{勘误表}\label{ux52d8ux8befux8868}}
\addcontentsline{toc}{chapter}{勘误表}

诗曰:

\begin{quote}
混沌未分天地乱,茫茫渺渺无人见。
自从盘古破鸿蒙,开辟从兹清浊辨。
覆载群生仰至仁,发明万物皆成善。
欲知造化会元功,须看《西游释厄传》。
\end{quote}

盖闻天地之数,有十二万九千六百岁为一元。将一元分为十二会,乃子、丑、
寅、卯、辰、巳、午、未、申、酉、戌、亥之十二支也。每会该一万八百岁。且就
一日而论:子时得阳气而丑则鸡鸣,寅不通光而卯则日出,辰时食后而巳则挨排,
日午天中而未则西蹉,申时晡而日落酉,戌黄昏而人定亥。譬于大数,若到戌会之
终,则天地昏而万物否矣。再去五千四百岁,交亥会之初,则当黑暗,而两间人
物俱无矣,故曰混沌。又五千四百岁,亥会将终,贞下起元,近子之会,而复逐渐
开明。邵康节曰``冬至子之半,天心无改移。一阳初动处,万物未生时'',到此,天
始有根;再五千四百岁,正当子会,轻清上腾,有日,有月,有星,有辰。日、月、
星、辰,谓之四象,故曰``天开于子''。又经五千四百岁,子会将终,近丑之会,而
逐渐坚实。《易》曰:大哉乾元,至哉坤元!万物资生,乃顺承天。至此,地始凝结。
再五千四百岁,正当丑会,重浊下凝,有水,有火,有山,有石,有土。水、火、
山、石、土,谓之五形,故曰``地辟于丑''。又经五千四百岁,丑会终而寅会之初,
发生万物,历曰``天气下降,地气上升;天地交合,群物皆生''。至此,天清地爽,
阴阳交合。再五千四百岁,正当寅会,生人,生兽,生禽,正谓天地人,三才定位。
故曰``人生于寅''。

\hypertarget{ux5e8fux8a00}{%
\chapter*{序言}\label{ux5e8fux8a00}}
\addcontentsline{toc}{chapter}{序言}

那猴在山中,却会行走跳跃,食草木,饮涧泉,采山花,觅树果;与狼虫为伴,
虎豹为群,獐鹿为友,猕猿为亲;夜宿石崖之下,朝游峰洞之中。真是``山中无甲
子,寒尽不知年。''

众猴都道:``这股水不知是那里的水。我们今日赶闲无事,顺涧边往上溜头寻看源流,耍子去耶!''喊一声,都拖男挈女,唤弟呼兄,一齐跑来,
顺涧爬山,直至源流之处,乃是一股瀑布飞泉。但见那:

一派白虹起,千寻雪浪飞。

海风吹不断,江月照还依。

冷气分青嶂,余流润翠微。

潺名瀑布,真似挂帘帷。

众猴拍手称扬道:``好水,好水!原来此处远通山脚之下,直接大海之波。''又道:``那
一个有本事的,钻进去寻个源头出来,不伤身体者,我等即拜他为王。''连呼了三声,
忽见丛杂中跳出一个石猴,应声高叫道:``我进去,我进去!''好猴!也是他:

今日芳名显,时来大运通。

有缘居此地,天遣入仙宫。

\hypertarget{abstract}{%
\chapter*{Abstract}\label{abstract}}
\addcontentsline{toc}{chapter}{Abstract}

众猴听得,个个欢喜。都道:``你还先走,带我们进去,进去!''石猴却又瞑目
蹲身,往里一跳,叫道:``都随我进来!进来!''那些猴有胆大的,都跳进去了;胆小
的,一个个伸头缩颈,抓耳挠腮,大声叫喊,缠一会,也都进去了。跳过桥头,一
个个抢盆夺碗,占灶争床,搬过来,移过去,正是猴性顽劣,再无一个宁时,只搬
得力倦神疲方止。

\hypertarget{ux7814ux7a76ux80ccux666f}{%
\chapter*{研究背景}\label{ux7814ux7a76ux80ccux666f}}
\addcontentsline{toc}{chapter}{研究背景}

美猴王领一群猿猴、猕猴、马猴等,分派了君臣佐使,朝游花果山,暮宿水帘
洞,合契同情,不入飞鸟之丛,不从走兽之类,独自为王,不胜欢乐。是以(R Core Team 2016):

春采百花为饮食,夏寻诸果作生涯。
秋收芋栗延时节,冬觅黄精度岁华。

美猴王享乐天真,何期有三五百载。一日,与群猴喜宴之间,忽然忧恼,堕下
泪来。众猴慌忙罗拜道:``大王何为烦恼?''猴王道:``我虽在欢喜之时,却有一点
儿远虑,故此烦恼。''众猴又笑道:``大王好不知足!我等日日欢会,在仙山福地,古
洞神洲,不伏麒麟辖,不伏凤凰管,又不伏人间王位所拘束,自由自在,乃无量之
福,为何远虑而忧也?''猴王道:``今日虽不归人王法律,不惧禽兽威严,将来年老
血衰,暗中有阎王老子管着,一旦身亡,可不枉生世界之中,不得久注天人之内?''
众猴闻此言,一个个掩面悲啼,俱以无常为虑。只见那班部中,忽跳出一个通背猿
猴,厉声高叫道:``大王若是这般远虑,真所谓道心开发也!如今五虫之内,惟有三
等名色,不伏阎王老子所管。''猴王道:``你知那三等人?''猿猴道:``乃是佛与仙与
神圣三者,躲过轮回,不生不灭,与天地山川齐寿。''猴王道:``此三者居于何所?''
猿猴道:``他只在阎浮世界之中,古洞仙山之内。''猴王闻之,满心欢喜,道:``我明
日就辞汝等下山,云游海角,远涉天涯,务必访此三者,学一个不老长生,常躲过
阎君之难。''噫!这句话,顿教跳出轮回网,致使齐天大圣成。众猴鼓掌称扬,都道:
``善哉!善哉!我等明日越岭登山,广寻些果品,大设筵宴送大王也。''(Xie 2016)

\hypertarget{ux65b9ux6cd5}{%
\chapter*{方法}\label{ux65b9ux6cd5}}
\addcontentsline{toc}{chapter}{方法}

这个是怎么回事,没有修改哪里啊

\hypertarget{ux7ed3ux679c}{%
\chapter*{结果}\label{ux7ed3ux679c}}
\addcontentsline{toc}{chapter}{结果}

猴王参访仙道,无缘得遇。在于南赡部洲,串长城,游小县,不觉八九年余。
忽行至西洋大海,他想着海外必有神仙。独自个依前作筏,又飘过西海,直至西牛
贺洲地界。登岸遍访多时,忽见一座高山秀丽,林麓幽深。他也不怕狼虫,不惧虎
豹,登山顶上观看。果是好山:
千峰排戟,万仞开屏。日映岚光轻锁翠,雨收黛色冷含青。瘦藤缠老树,古渡
界幽程。奇花瑞草,修竹乔松:修竹乔松,万载常青欺福地;奇花瑞草,四时不谢
赛蓬瀛。幽鸟啼声近,源泉响溜清。重重谷壑芝兰绕,处处崖苔藓生。起伏峦头
龙脉好,必有高人隐姓名。
正观看间,忽闻得林深之处,有人言语,急忙趋步,穿入林中,侧耳而听,原来是
歌唱之声。歌曰:
``观棋柯烂,伐木丁丁,云边谷口徐行。卖薪沽酒,狂笑自陶情。苍径秋高,
对月枕松根,一觉天明。认旧林,登崖过岭,持斧断枯藤。
收来成一担,行歌市上,易米三升。更无些子争竞,时价平平。不会机谋巧算,
没荣辱,恬淡延生。相逢处,非仙即道,静坐讲《黄庭》。''
美猴王听得此言,满心欢喜道:``神仙原来藏在这里!''即忙跳入里面,仔细再看,
乃是一个樵子,在那里举斧砍柴。但看他打扮非常:
头上戴箬笠,乃是新笋初脱之箨;身上穿布衣,乃是木绵拈就之纱;腰间系环
绦,乃是老蚕口吐之丝;足下踏草履,乃是枯莎槎就之爽。手执钢斧,担挽火麻
绳;扳松劈枯树,争似此樵能!

\hypertarget{ux8ba8ux8bba}{%
\chapter*{讨论}\label{ux8ba8ux8bba}}
\addcontentsline{toc}{chapter}{讨论}

这里也是可以不是占位符号的

\hypertarget{ux7ed3ux8bba}{%
\chapter*{结论}\label{ux7ed3ux8bba}}
\addcontentsline{toc}{chapter}{结论}

这个地方商业

\hypertarget{ux53c2ux8003ux6587ux732e}{%
\chapter*{参考文献}\label{ux53c2ux8003ux6587ux732e}}
\addcontentsline{toc}{chapter}{参考文献}

fasdf

\hypertarget{refs}{}
\leavevmode\hypertarget{ref-R-base}{}%
R Core Team. 2016. \emph{R: A Language and Environment for Statistical Computing}. Vienna, Austria: R Foundation for Statistical Computing. \url{https://www.R-project.org/}.

\leavevmode\hypertarget{ref-R-bookdown}{}%
Xie, Yihui. 2016. \emph{Bookdown: Authoring Books and Technical Documents with R Markdown}. Boca Raton, Florida: Chapman; Hall/CRC. \url{https://github.com/rstudio/bookdown}.
	\backmatter
	{\color{TEXTColor}
	\end{document}
